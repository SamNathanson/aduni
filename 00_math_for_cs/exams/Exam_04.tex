\documentclass[10pt]{amsart}
 
\setlength{\parsep}{3pc}
\setlength{\itemsep}{0.2in}


\usepackage{fullpage}
\usepackage{psfig}



\newcommand{\Z}{\mathbb Z}
\newcommand{\F}{\mathbb F}
\newcommand{\R}{\mathbb R}
\newcommand{\C}{\mathbb C}
\newcommand{\N}{\mathbb N}
\newcommand{\Q}{\mathbb Q}
%\newcommand{\to}{\rightarrow}

\newtheorem{thm}{Theorem}[section] 
\newtheorem{theorem}[thm]{Theorem} 
\newtheorem{corollary}[thm]{Corollary} 
\newtheorem{lemma}[thm]{Lemma} 
\newtheorem{prop}[thm]{Proposition} 
\newtheorem{definition}[thm]{Definition} 
\newtheorem{remark}[thm]{Remark}  
\newtheorem{fact}{Fact}[section]

\newenvironment{EG}[1]{{\vspace{1 ex}}\noindent {\sc Example.}{#1}{\hfill{$\diamondsuit$}}\\{\vspace{1 ex}}}



\title[\hskip 0.2in Final Exam\hfill Name:\hskip 2in]{Exam 4:  Final Exam}

\begin{document}

%\renewcommand{\arraystretch}{2}

\begin{figure}[h]
\centerline{
\psfig{figure=logo.ps}
}
\end{figure}

\centerline{\Large{\sc{Month 0: Mathematics for Computer Science}}}
\medskip

\maketitle

\vfill


\centerline{\LARGE{September 29, 2000}}

\vskip 1in

\hskip 2in\Large{Name:}

\vskip 1in

\noindent You may  not consult any books or papers.  You may use a
calculator (or the calculator on your computer), but you may not use
any other computing or graphing device other than your own head!

Numerical answers should be integers, unless otherwise indicated!

Show all your work.  A (correct) numerical answer with no work indicated
may not get full credit.  An incorrect answer with accompanying work may
earn partial credit.

\vfill\pagebreak


\begin{enumerate}

\begin{center}
{\sc Calculus}
\end{center}


\item  {\sc Short Answer} (Three parts.  50 points.) Suppose you are
given a differentiable function $f(x)$. 
\begin{enumerate}
\item If $f'(0)=1$, what does this tell you about the original
function $f(x)$?
\vfill
\item If $\int_{-1}^1f(x)\ dx=4$, what does this tell you about the
original function?
\vfill
\item If the graph of $f(x)$ from $x=2$ to $x=4$ is the graph in the
figure below, what does the integral
$$
\int_2^4\pi(f(x))^2\ dx
$$
compute?

\begin{figure}[h]
\flushright{
\parbox{3in}{
\psfig{figure=function4.ps,width=2in}
}}
\flushright{
\parbox{3in}{
\caption[square]{This shows the graph of the function $f(x)$ from
$x=2$ to $x=4$.}
}
}
\end{figure}
\end{enumerate}
\pagebreak

\item {\sc Pennies from heaven.}  (Three parts.  50 points.)
\begin{enumerate}
\item An office worker on the eighth floor
of the Hancock building decides to throw a penny out of his window to
see how big of a dent it will make in the sidewalk.  We know that
acceleration is $-32\ ft/s^2$, where $ft$ is feet and $s$ is seconds.
(The negative sign signals that acceleration is {\em down} towards the 
ground.)  If he throws the penny upwards with an initial velocity of
$18\ ft/s$, and his office window is $60$ feet off the ground, what is
the function describing the height of the penny in terms of the number 
of seconds since the penny was thrown? ({\sc Hint:} When you integrate
the acceleration to get the velocity, remember to add in the
additional information to get your constant of integration! etc.)
\vfill
\item Suppose an office worker a couple floors down also throws a 
penny.  This
time, the function describing the height of the function is
$f(t)=-16t^2+32t+48$.  (Don't worry if this doesn't match your answer 
in the first part -- it should not!) Graph this function of the
height.  (Graph paper on the next page.) When does
this second penny hit the ground?  What is its velocity when it hits the
ground?


\vfill\pagebreak

\begin{figure}[h]
\begin{center}
\psfig{figure=graphPaper.ps}
\end{center}
\end{figure}



\vfill\pagebreak


\item Unbeknownst to the second penny-thrower on the sixth floor, there is
a
sharp-shooting penny-thrower on the twelfth floor.  When she sees this
second penny-thrower
starting to throw his penny, she pulls one out of her own.  She throws 
her penny at exactly the same time with an initial velocity of $16\ ft/s$
{\em downwards}, and throws it from a height of $96\ ft$. What function
describes the height of this third penny? At what time after the second 
and third pennies are thrown are those two pennies at
the same height (thereby potentially colliding midair if the sharpshooter
is any good!)?
\end{enumerate}

\vfill\pagebreak

\item {\sc Moving day.} (One part. 40 points.) Erica buys a $30$
inch by $30$ inch piece of corrugated cardboard, and uses it to make a
box with a lid.  She cuts out the corners (and throws them away!), and
folds the following shape along the dotted lines to make the box.
What are the dimensions of the box made from this cardboard that has
maximum volume?

\begin{figure}[h]
\flushright{
\parbox{3in}{
\psfig{figure=box.ps,width=2in}
}}
\flushright{
\parbox{3in}{
\caption[square]{This is a rough sketch of the box Erica is going to
make.  She throws away the shaded portions of the poster board, and
folds along the dotted lines. (Not to scale!!)}
}
}
\end{figure}

\vfill\pagebreak

\item {\sc Buoy in the harbor.} (2 parts.  40 points.)

\begin{enumerate}
\item Consider the function $y=3x$.  Rotate the region enclosed by
this function, $y\leq 3$ and $x\geq 0$, around the $y$-axis.  The
resulting solid is a cone.  Compute its volume.


\begin{figure}[h]
\flushright{
\parbox{3in}{
\psfig{figure=cylGraph.ps,width=2in}
}}
\flushright{
\parbox{3in}{
\caption[square]{This is a rough sketch of the region you are to
rotate around the $y$-axis.}
}
}
\end{figure}

\vfill\pagebreak


\item  You have computed the volume of a cone of height $3$ with 
radius $1$.  Suppose this cone is
weighted at the tip, and is floating as a buoy in the water (see
figure).  The cone is weighted in such a way that $\frac{1}{4}$ of its 
volume is below the surface of the water.  If $x$ is the height of the 
portion below the water, then the volume of the portion below the
water is
$$
V_x=\frac{\pi x^3}{27}.
$$
If $V$ is the total volume of the cone (which you computed above),
then use Newton's method to solve the equation
$$
\frac{\pi x^3}{27}=\frac{V}{4}\ \longrightarrow\ \frac{\pi x^3}{27}-\frac{V}{4}=0
$$
for the zero.  This $x$ will be the height of the portion under the
water.  Do two (2) iterations of Newton's method ({\sc Hint:} the root 
is between $1$ and $2$.  You will not get an integer answer!)

\begin{figure}[h]
\flushright{
\parbox{3in}{
\psfig{figure=buoy.ps,height=2in}
}}
\flushright{
\parbox{2.75in}{
\caption[square]{This is a figure showing the buoy.  The shaded
portion represents the part of the cone which is underwater.  The
volume of this portion is $V_x=\frac{\pi x^3}{27}$.}
}
}
\end{figure}

\vfill\pagebreak



\end{enumerate}

\begin{center}
{\sc Linear Algebra}
\end{center}



\item {\sc Short Answer.} (3 parts.  50 points.) Consider the following
system of equations and corrseponding coefficient matrix.
$$
\left\{\begin{array}{ccccccccccc}
2x_1 & + & 2x_2&+&2x_3&+&6x_4&+&14x_5&+&22x_6\\
x_1 & + & x_2 & - & x_3 & - & x_4 & - &  x_5 & - & 3x_6\\
 &  & &  & 2x_3 & +& 3x_4 & + &  7x_5 & + & 10x_6\\
-x_1 & - & x_2 & - & x_3 & - & 4x_4 & - &  8x_5 & - & 13x_6\\
 &  &  &  &  & & 2x_4 & + &  2x_5 & + & 11x_6
\end{array}\right.
$$
\vskip 0.5in
$$
A=\left[\begin{array}{rrrrrr}
2&2&2&6&14&22\\
1&1&-1&-1&-1&-3\\
0&0&2&3&7&10\\
-1&-1&-1&-4&-8&-13\\
0&0&0&2&2&11
      \end{array}\right]
$$

\begin{enumerate}
\item Just by looking at this system of equations, can you tell
whether or not this will have a {\em unique} solution to
$$
A\cdot x=\left[\begin{array}{c}1\\2\\3\\4\\5\\6\\7\end{array}\right]?
$$
\vfill\pagebreak
\item The reduced row echelon form of $A$ is
$$
R=\left[\begin{array}{rrrrrr}
1&1&0&0&2&0\\
0&0&1&0&2&0\\
0&0&0&1&1&0\\
0&0&0&0&0&1\\
0&0&0&0&0&0
	\end{array}\right].
$$
\begin{enumerate}
\item What are the pivot and free variables?  What is the rank of $A$?
\vfill
\item What does the row of all zeroes mean?
\vfill
\item Explain the following statement.  There are either no solutions or
infinitely many solutions to the equation
$$
A\cdot x=b.
$$
\end{enumerate}
\end{enumerate}



\vfill\pagebreak

\item {\sc Planes, trains and automobiles.} (1 part. 30 points.)
Suppose a paint company paints automobiles, trains and planes.  Each
automobile takes 10 man hours to prepare, 30 man hours to paint, and 12
man hours to add finishing touches (the painters are quite
meticulous).  Each train takes 20 man hours to prepare, 75 man hours to
paint, and 36 man hours to add finishing touches.  Each plane takes 40
man hours to prepare, 135 man hours to paint, and 64 man hours to add
finishing touches.  If the paint company decides to use 760 man hours
towards preparation, 2595 man hours towards painting, and 1224 man hours
towards finishing touches each week, how many planes, trains and
automobiles do they paint each week?

\vfill\pagebreak

\item {\sc Rabbits, rabbits everywhere!}  (2 parts.  40 points.)
\begin{enumerate}
\item Consider the Fibonacci problem with a minor adjustment.  You start in
month zero with a pair of newborn rabbits.  After one month, they
begin to reproduce and each subsequent month they give birth to $2$
new pairs of rabbits.  This gives us the recurrence relation
$$
F_n=F_{n-1}+2F_{n-2},
$$
with $F_0=0$ and $F_1=1$.  In this case, we can write this in matrix
form as
$$
\left[\begin{array}{cc}1&2\\1&0\end{array}\right]
\cdot\left[\begin{array}{c}F_{n+1}\\F_n\end{array}\right]
=\left[\begin{array}{c}F_{n+2}\\F_{n+1}\end{array}\right].
$$
Diagonalize the matrix
$$
A=\left[\begin{array}{cc}1&2\\1&0\end{array}\right]
$$
and write it as $A=B\cdot D\cdot B^{-1}$, where $D$ is a diagonal
matrix. ({\sc Hint:} The diagonal matrix has integer entries on the diagonal!)
You could then solve for a closed formula for $F_n$, but you do not have
to here!


\vfill
\end{enumerate}


\end{enumerate}






\end{document}
