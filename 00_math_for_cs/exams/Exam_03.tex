\documentclass[10pt]{amsart}
 
\setlength{\parsep}{3pc}
\setlength{\itemsep}{0.2in}


\usepackage{fullpage}
\usepackage{psfig}



\newcommand{\Z}{\mathbb Z}
\newcommand{\F}{\mathbb F}
\newcommand{\R}{\mathbb R}
\newcommand{\C}{\mathbb C}
\newcommand{\N}{\mathbb N}
\newcommand{\Q}{\mathbb Q}
%\newcommand{\to}{\rightarrow}

\newtheorem{thm}{Theorem}[section] 
\newtheorem{theorem}[thm]{Theorem} 
\newtheorem{corollary}[thm]{Corollary} 
\newtheorem{lemma}[thm]{Lemma} 
\newtheorem{prop}[thm]{Proposition} 
\newtheorem{definition}[thm]{Definition} 
\newtheorem{remark}[thm]{Remark}  
\newtheorem{fact}{Fact}[section]

\newenvironment{EG}[1]{{\vspace{1 ex}}\noindent {\sc Example.}{#1}{\hfill{$\diamondsuit$}}\\{\vspace{1 ex}}}



\title[\hskip 0.2inExam 3\hfill Name:\hskip 2in]{Exam 3:  Linear algebra}

\begin{document}

%\renewcommand{\arraystretch}{2}

\begin{figure}[h]
\centerline{
\psfig{figure=logo.ps}
}
\end{figure}

\centerline{\Large{\sc{Month 0: Mathematics for Computer Science}}}
\medskip

\maketitle

\vfill


\centerline{\LARGE{September 24, 2000}}

\vskip 1in

\hskip 2in\Large{Name:}

\vskip 1in

\noindent You may  not consult any books or papers.  You may use a
calculator (or the calculator on your computer), but you may not use
any other computing or graphing device other than your own head!

\vfill\pagebreak


\begin{enumerate}

\item Consider the following matrices and vectors.
$$
\begin{array}{ccc}
A=\left[\begin{array}{rrr}1&3&-4\\2&-2&0\\3&-1&0\end{array}\right] &
B=\left[\begin{array}{rrr}-2&1&3\\1&4&0\end{array}\right] &
C=\left[\begin{array}{rr}1&2\\0&3\\-1&4\end{array}\right]
\end{array}
$$
$$
\begin{array}{cc}
v=\left[\begin{array}{r}1\\3\\-1\end{array}\right] &
w=\left[\begin{array}{r}-1\\6\end{array}\right]
\end{array}
$$
What are all of the matrix multiplications (of two matrices or one matrix
and one vector) that you can do?  Do two (2) of these multiplications
(your choice which ones).

\vfill\pagebreak


\item Consider the following matrices.
$$
\begin{array}{cc}
A=\left[\begin{array}{rrrr}1&2&-1&0\\0&1&0&1\\
			   1&2&0&1\\0&2&1&-4\end{array}\right]  &
B=\left[\begin{array}{rrr}1&-1&2\\2&0&1\\2&3&-3\end{array}\right] 
\end{array}
$$
\begin{enumerate}
\item Compute $\det(A)$.
\vfill
\item Compute $B^{-1}$, the inverse of $B$.
\end{enumerate}



\vfill\pagebreak

\item Find the complete solution to the following linear system of
equations.
$$
\begin{array}{cccc}
	\left[\begin{array}{rrrrr}
		1&2&-1&-2&1\\
		1&2&0&0&3\\
		2&4&1&2&9
	\end{array}\right] &
	\left[\begin{array}{c}x_1\\x_2\\x_3\\x_4\\x_5\end{array}\right] &
	= &
	\left[\begin{array}{r}0\\-1\\-4\end{array}\right]
\end{array}
$$




\vfill\pagebreak

\item Which of the following are vector spaces?  Why or why not?

\begin{enumerate}
\item For an $m\times n$ matrix $A$, the nullspace $N(A)=\{ v\in\R^n\
|\ A\cdot v=0\}$.
\vfill
\item $V=\{ (x_1,x_2,x_3)\ |\ x_1,x_2,x_3\in\R\mbox{ and }
x_1+x_2+x_3=0\}$.
\vfill
\item $\mathcal{F}_3=\{ f:\R\to\R\ |\ f(0)=3\}$.
\vfill
\item $W=\{ (x,|x|)\ |\ x\in\R\}$.
\end{enumerate}


\vfill\pagebreak

%\item Factor the matrix
%$$
%\left[\begin{array}{rrr}1&0&3\\2&1&4\\0&2&6\end{array}\right],
%$$
%into the form $A=L\cdot U$, the product of a lower triangular matrix
%$L$ and an upper triangular matrix $U$.
%
%\vfill\pagebreak

\item At a local movie theater, three groups of people are in line to
buy tickets.  The first group buys three adult tickets and three child 
tickets, for \$39.  The second group buys three adult tickets and four 
senior tickets for \$44.  The third group has just come from the
restaurant next door.  They buy two discount tickets and two child
tickets for \$22.  If senior tickets are the same price as child
tickets, then how much does each type of ticket cost?


\end{enumerate}


\vfill\pagebreak

\noindent {\sc{Extra problem.}} Only do this
problem if you have completed and checked over your exam, and feel like
taking a look at it! It will not count
towards your exam grade, but is meant to be a fun problem to think
about!

Given four points $P_1$, $P_2$, $P_3$ and $P_4$ in the plane ($\R^2$), 
there are six distances between pairs of points (i.e. the distance
from $P_1$ to $P_2$, from $P_1$ to $P_3$, etc.).  How many ways are
there of placing the points in the plane so that there are exactly two 
distinct distances among the six?  An example of such is the square
shown in the figure below.  (Two identical pictures with the vertices
relabelled do not count as distinct pictures!)


\begin{figure}[h]
\flushright{
\parbox{3in}{
\psfig{figure=square.ps,width=2in}
}}
\flushright{
\parbox{3in}{
\caption[square]{This show the square.  The two distinct distances are
the side length and the diagonal length.}
}
}
\end{figure}









\end{document}
