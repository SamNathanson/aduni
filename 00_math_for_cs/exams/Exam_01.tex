\documentclass[10pt]{amsart}
 
\setlength{\parsep}{3pc}
\setlength{\itemsep}{0.2in}


\usepackage{fullpage}
\usepackage{psfig}



\newcommand{\Z}{\mathbb Z}
\newcommand{\F}{\mathbb F}
\newcommand{\R}{\mathbb R}
\newcommand{\C}{\mathbb C}
\newcommand{\N}{\mathbb N}
\newcommand{\Q}{\mathbb Q}
%\newcommand{\to}{\rightarrow}

\newtheorem{thm}{Theorem}[section] 
\newtheorem{theorem}[thm]{Theorem} 
\newtheorem{corollary}[thm]{Corollary} 
\newtheorem{lemma}[thm]{Lemma} 
\newtheorem{prop}[thm]{Proposition} 
\newtheorem{definition}[thm]{Definition} 
\newtheorem{remark}[thm]{Remark}  
\newtheorem{fact}{Fact}[section]

\newenvironment{EG}[1]{{\vspace{1 ex}}\noindent {\sc Example.}{#1}{\hfill{$\diamondsuit$}}\\{\vspace{1 ex}}}



\title[\hskip 0.2inExam 1\hfill Name:\hskip 2in]{Exam 1:  Differential calculus}

\begin{document}

\begin{figure}[h]
\centerline{
\psfig{figure=logo.ps}
}
\end{figure}

\centerline{\Large{\sc{Month 0: Mathematics for Computer Science}}}
\medskip

\maketitle

\vfill


\centerline{\LARGE{September 9, 2000}}

\vskip 1in

\hskip 2in\Large{Name:}

\vskip 1in

\noindent You may consult your paper containing trigonometric
identities, but you may not consult any other books or papers.  You
may use a calculator (or the calculator on your computer), but you may
not use any other computing or graphing device other than your own
head! 

\vfill\pagebreak
\begin{enumerate}

\item Differentiate the following functions.
\begin{enumerate}
\item $f(x)=\cos(x^3+4)$
\vfill
\item $f(x)=e^{x}+3x^2+7$
\vfill
\item $f(x)=(x^2+2x)^{50}\cdot(x-3)$
\end{enumerate}

\vfill\pagebreak

\item Graph the following function, using the first and second
derivatives: 
$$
f(x)=\frac{x^3-3x^2+2x-6}{x-3}=\frac{(x-3)\cdot(x^2+2)}{(x-3)}.
$$

\vfill 

\begin{figure}[h]
\psfig{figure=graphPaper.ps,width=5in}
\end{figure}



\pagebreak

\item A farmer wants to fence off a rectangular garden next to his
barn, using the barn as one of the walls. (See figure below.)  He goes
to Agway, and buys 240m of fencing.  What is the biggest garden he can 
fence off?

\begin{figure}[h]
\hfill\psfig{figure=barn.ps,width=2in}
\end{figure}

\vfill\pagebreak

\item When Mike Allen (the kicker for his MIT intramural football
team) kicks the football,
the ball goes up in the air and  reaches a height of
$s(t)=2t-\frac{t^2}{8}$ meters after $t$ seconds. (See figure below
for a rough sketch of this.)

\begin{figure}[h]
\centerline{
\psfig{figure=football.ps,width=3in}
}
\end{figure}

\begin{enumerate}
\item What is the velocity of the ball when $t=2$seconds?
\vfill
\item What is the maximum height of the ball?
\vfill
\item What is the acceleration of the ball at $t=4$seconds?
\end{enumerate}


\vfill\pagebreak

\noindent {\sc{Extra problem -- A maximization puzzler.}} Only do this
problem if you completed and checked over your exam, and feel like
taking a look at this!  It will not count
towards your exam grade, but is meant to be a fun problem to think
about!

For your birthday, the king decides to give you as much land as you can
claim in one day.  He gives you some wooden posts.  You are free to place
these posts in the ground wherever you want, and at the end of the day the
land contained in the convex hull of the posts will be yours. (That
is, the king will send out men to wrap string around all the posts you 
have planted in the ground, and you get to keep the land inside.)

\begin{itemize}
\item You have exactly 24 hours (1440 minutes).
\item It takes you exactly 1 minute to pound a post into the ground.
\item You walk at at a constant speed.
\item You must end to your initial starting point.
\end{itemize}
What do you do? (We are assuming, of course, that you want to get as
much land as possible!  The full solution to this problem probably
requires the use of a computer to find the exact answer.  Try to write 
down the function that you are trying to maximize, though.)


\end{enumerate}





\end{document}
