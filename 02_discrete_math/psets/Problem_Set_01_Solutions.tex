%LaTeX
\documentclass[12pt]{amsart}
\usepackage{fullpage}
\usepackage{amsmath}
\newcommand{\tab}{\makebox[4em]{}}
\newcommand{\ov}[1]{\overline{#1}}
\newcommand{\ra}{\rightarrow}
\newcommand{\lra}{\leftrightarrow}




% ----------------
% PARAGRAPH INDENTATION
\setlength{\parindent}{0in}

% SPACE BETWEEN PARAGRAPHS
\setlength{\parskip}{\medskipamount}

% ----------------------------------------------------------------
% LISP CODE DISPLAYS.
% Lisp code displays are enclosed between \bid and \eid.
% Most characters are taken verbatim, in typewriter font,
% Except:
%  Commands are still available (beginning with \)
%  Math mode is still available (beginning with $)

\outer\def\beginlisp{%
  \begin{minipage}[t]{\linewidth}
  \begin{list}{$\bullet$}{%
    \setlength{\topsep}{0in}
    \setlength{\partopsep}{0in}
    \setlength{\itemsep}{0in}
    \setlength{\parsep}{0in}
    \setlength{\leftmargin}{1.5em}
    \setlength{\rightmargin}{0in}
    \setlength{\itemindent}{0in}
  }\item[]
  \obeyspaces
  \obeylines \footnotesize\tt}

\outer\def\endlisp{%
  \end{list}
  \end{minipage}
  }

{\obeyspaces\gdef {\ }}

\begin{document}
\begin{center}
\textbf{ArsDigita University}

\textbf{Month 2: Discrete Mathematics - Professor Shai Simonson}

\textbf{Problem Set 1 -- ANSWERS}

By: Dimitri Kountourogiannis

\end{center}


\begin{enumerate}
\item { Logic Proofs }
    \begin{enumerate}
    \item   Prove that $a \rightarrow b$
            is equivalent to $\neg b
            \rightarrow \neg a$
            using a truth table. \\
    \noindent {\bf  ANSWER:}
    We compute the truth table values and observe they are the same.
    Note that $\neg b \rightarrow \neg a$ is called the contrapositive
    of $a \rightarrow b$.
    (Check out \\ http://logik.phl.univie.ac.at/$\sim$chris/formular-uk-zentral.html \\
    for an automated truth-table generator)
    \[
    \begin{array}{c|c|c|c|c|c}
        a&b&\neg a&\neg b&a\rightarrow b&(\neg b) \rightarrow (\neg a)\\\hline
        T   &   T   &   F   &   F   &   T   &   T\\
        T   &   F   &   F   &   T   &   F   &   F\\
        F   &   T   &   T   &   F   &   T   &   T\\
        F   &   F   &   T   &   T   &   T   &   T
    \end{array}
    \]
    \item   Prove it using {\em algebraic} identities. \\
    \noindent {\bf  ANSWER:}
    Using the equivalence rules on pages 17-18 of Rosen,
        \begin{align*}
            a \rightarrow b & \Leftrightarrow \neg a \wedge b \\
                            & \Leftrightarrow b \wedge \neg a \\
                            & \Leftrightarrow (\neg \neg b) \wedge \neg a\\
                            &\Leftrightarrow \neg b \rightarrow \neg a
        \end{align*}
    \item   Prove that $a \rightarrow b$ is not
            equivalent to $ b \rightarrow a$.\\
    \noindent {\bf  ANSWER:}
    This is the sort of thing that is easy to show with a truth table (semantically)
    and is extremely difficult to show using the algebraic identities (syntactically).
    If $a \rightarrow b$ were equivalent to $b \rightarrow a$, then the two would have
    to agree for all possible {\em true/false} assignments of $a$ and $b$. But this is not
    the case: if we let $a$ be {\em true} and $b$ be {\em false}, then $a
\rightarrow b$
    evaluates to {\em false}, and $b \rightarrow a$ evaluates to {\em true}. So the two
    formulas are not equivalent.
    \end{enumerate}
    \vskip 5pt
\item {Aristotle's Proof that the Square Root of Two is Irrational.}\\
    \begin{enumerate}
    \item Prove the  {\em lemma,} used by Aristotle in his proof,
        which says that if $n^{2}$ is even, so is
        $n$. (Hint: Remember that $a \rightarrow  b$
        is equivalent to $\neg b \rightarrow \neg a$.\\
    \noindent {\bf  ANSWER:}
    It is easiest to prove the contrapositive of the lemma, that if
    $n$ is odd, then $n^2$ is odd. An odd number is one that can be
    written in the form of $2k+1$ for some other whole number $k$.
    So suppose that $n = 2k+1$. Then
    \[
    n^2 = (2k+1)^2 = 4k^2 + 4k + 1 = 2(2k^2 + 2k) + 1
    \]
    And so $n^2$ is odd. This proves the contrapositive and hence the lemma.
    \item
        Prove that the square root of 3 is irrational using Aristotle's
        techniques. Make sure to prove the appropriate lemma.\\
    \noindent {\bf  ANSWER:}
    We need the following lemma: \\

    Lemma: \\
    {\em For any whole number $n$, $n^2$ is divisible by $3$ implies that
    $n$ is divisible by $3$ } \\

    Proof of Lemma: Again it is easier to prove the contrapositive:\\

    {\em $n$ is not divisible by $3$ implies that $n^2$ is not divisible by 3.} \\

    So assume that $n$ is not divisible by $3$ then $n$ is of the form
    $n= 3k+1$, or $n=3k+2$ for some whole number $k$. Then in the first case
    \[
    n^2 = 9k^2 +6k + 1 = 3(3k^2 + 2k) + 1
    \]
    which is not divisible by $3$. The second case is similar. Q.E.D. Lemma. \\

    Proof that the square root of $3$ is irrational: \\
    Suppose not, and that we can write
    \[
    \sqrt{3} = \frac{p}{q}
    \]
    with $\frac{p}{q}$ in lowest terms. Then
    \[
    3q^2 = p^2
    \]
    And so $p^2$ is divisible by $3$. By the lemma it follows that $p$
    is divisible by $3$, say $p = 3p_1$ for some whole number $p_1$.
    Plugging back in to the formula, we get
    \[
    q^2 = 3 p_1^2
    \]
    Now the tables are turned! $q^2$ is divisible by $3$ and so
    by the lemma $q$ is divisible by $3$, say $q = 3 q_1$ for some whole
    number $q_1$. This implies that
    \[
    \frac{p}{q} = \frac{3p_1}{3q_1} = \frac{p_1}{q_1}
    \]
    This is the contradiction we were looking for: We assumed that
    $\frac{p}{q}$ was in lowest terms, which we can always do with a fraction,
    but the assumption that
    $\frac{p}{q} = \sqrt{3}$ implies that $\frac{p}{q} $ is {\em not} in lowest
    terms. Contradiction! Therefore $\sqrt{3}$ is not a rational number.
    Which is quite a disturbing thing, if you're an ancient Greek.
    \item
        If we use Aristotle's technique to  {\em prove} the untrue assertion
        that the square root of 4 is irrational, where  {\em exactly} is
        the hole in the proof?\\
\noindent {\bf  ANSWER:}
    The problem is that when we reach the equation
    \[
    q^2 = 4 p_1^2
    \]
    We {\em cannot} use the fact that $4$ divides $q^2$ to conclude that
    $4$ divides $q$. This is just false. For example: $2^2 = 4$, or $6^2 = 36$.
    \item Using the fact that the square root of two is irrational, prove
        that $\sin (\frac{\pi}{4})$ is irrational.\\
    \noindent {\bf  ANSWER:}
    There is no way that $\sin (\frac{\pi}{4}) = \frac{1}{\sqrt{2}}$ could be
    equal to a rational number $\frac{p}{q}$ for then it would follow
    that $\sqrt{2}$ would equal the inverse, $\frac{q}{p}$, which we have shown
    to be impossible.
    \end{enumerate}
\item
    In ADU-ball, you can score 11 points for a goal, and 7 for a
    near miss.
    \begin{enumerate}
    \item
    Write a Scheme program that prints out the number of goals and
    the number of near misses to achieve a given total greater than
    60.\\
    \item Prove that you can achieve any score greater than 60. Think
    inductively and experiment.\\
    \end{enumerate}
    \noindent {\bf  ANSWER:}
    \begin{enumerate}
    \item
    ADU-BALL returns a list of all goal, near-miss  pairs that add up to the given number
    The loop subtracts off multiples of $7$ from the given number,
    and adds a solution to the list if the result is divisible by
    11. Note the use of let*, which unlike let,
    does perform each assignment before going to the next one.
\beginlisp
(define (adu-ball n)
 (define (iter i result)
  (let* ((a (- n (* 7 i))) (b (remainder a 11)))
   (cond ((< a 0) result)
         ((= b 0) (iter (inc i) (cons (cons (/ a 11) i) result)))
         (else (iter (inc i) result)))))
 (iter 0 nil))
\endlisp \vskip 10pt

\beginlisp
(adu-ball 187)
;Value: ((3 . 22) (10 . 11) (17 . 0))
\endlisp \vskip 10pt
    \item
    Actually you can achieve any score greater than or equal to $60$.
    We will prove this by induction. There will actually be seven base cases
    here, since the induction will go down by steps of seven. Here is a formal
    statement of what will prove:

    For any $n \ge 60$ there are natural numbers $x,y$ such that
    $ n = 11 x + 7 y$

    Base Cases:
    \begin{itemize}
    \item $n=60$. We can let $x=1$, $y=7$ \\
    \item $n=61$. We can let $x=3$, $y=4$ \\
    \item $n=62$. We can let $x=5$, $y=1$ \\
    \item $n=63$. We can let $x=0$, $y=9$ \\
    \item $n=64$. We can let $x=2$, $y=6$ \\
    \item $n=65$. We can let $x=4$, $y=3$ \\
    \item $n=66$. We can let $x=6$, $y=0$ \\
    \end{itemize}

    Induction step. Suppose $67 \le n$ Suppose that we can express any number $k$
    $60 \le k < n$. We would like to show that there is
    a $u$ and a $v$ such that $n = 11 u + 7 v$.
    This isn't so bad. We know that $60 \le n - 7 < n$, so by induction.
    We can find an $x$ and a $y$ such that.
    \[
    11 x + 7 y = n - 7
    \]
    This implies that
    \[
    11 x + 7( y + 1) = n
    \]
    and so we can let $u = x$ and $v = y +1$ and our induction
    step is proven.
    Q.E.D.
    \end{enumerate}
\item Prove by induction that there are 2$^{\mathit{n}}$ possible
      rows in a truth table with  $ n$ variables.\\
    \noindent {\bf  ANSWER:}\\
    Base Case: When $n= 1$, we have $2$ rows, which equals $2^1$\\
    Induction Step: Assume that an $n$-variable truth table has
    $2^n$ rows. We want to show that an $(n+1)$-variable truth
    table has $2^{n+1}$ rows. Well the way we construct a $(n+1)$
    variable truth-table from an $n$-variable truth table is we
    take two copies of it and we set the new variable equal
    to $1$ in the first copy and we set the new variable equal
    to $0$ in the second copy. So the total number of rows
    is
    \begin{align*}
    \text{number of rows in copy-1} + \text{number of rows in copy-2}
    &= 2^n + 2^n \\
    &= 2^{n+1}
    \end{align*}
    So our induction step is proven. Q.E.D.

\item   In the restroom of a fancy Italian restaurant in Mansfield,
        MA, there is a sign which reads:  {\em Please do not leave
        valuables or laptop computers in your car.}\\
        Assuming that a laptop computer is considered a valuable, prove
        using formal logic, that the sentence  {\em Please do not leave
        valuables in your car} is equivalent to the sign in the restroom.
        Prove that  {\em Please do not leave laptops in your car} is not
        equivalent. \\
        \noindent {\bf  ANSWER:}

        This is a bit confusing at first glance, since our logical
        system isn't designed to deal with imperatives. If we
        ignore that little difficulty, there are many ways to turn the above sentences
        into formulas. Here is one:
        Let $V$ be the statement there are valuables in the car.
        Let $L$ be the statement there are laptops in the car.
        We know that
        \[
        L \ra V
        \]
        We want to show that the above assumption
        implies that
        \[
        L \vee V \Leftrightarrow V
        \]
        The easiest way is to compare the truth-table values of
        $L \vee V$ and $V$ in the three cases where $L \ra V$ holds
        \[
        \begin{array}{c|c|c|c}
        L   &   V   &   L \ra V &  L \vee V \\\hline
        0   &   0   &   1       &   0       \\
        0   &   1   &   1       &   1       \\
        1   &   1   &   1       &   1
    \end{array}
    \]
        We see that in the cases where $L \ra V$ holds,
        $V$ and $L \vee V$ are equivalent to each other
        and not equivalent to $L$
\item   Prove that $a | b$, $a$
        nand $b$, which is defined to be
    $\neg (a \wedge b)$, is
    complete. Write
    $(~a~\rightarrow~b)~\rightarrow~b$
    using just $|$ (nand), then using just
    $\downarrow$ (nor).\\
    \noindent {\bf  ANSWER:}
    \begin{enumerate}
    \item
        We just need to construct not, and from nand.
        Negation is very easy:
        \[
        a \vert a = \overline{ a \cdot a } \text{, which is equivalent to } \overline{a}
        \]
        Now $a \cdot b$ is equivalent to  $\overline{a \vert b}$,
        which by the above is equivalent to
        \[
        (a \vert b) \vert (a \vert b)
        \]
        Therefore nand is complete, since or, and are complete.
    \item
        One way to do this is to simplify
        \[
        ( a \rightarrow b) \rightarrow b
        \]
        Plugging in the definition of implication, we get,
        \[
         \overline{ (\overline{a} + b)} + b
        \]
        Which is equivalent to
        \[
        a\cdot \overline{b} + b
        \]
        Distributing the or over the and, this is equivalent to
        \[
        (a+ b)(\overline{b} + b)
        \]
        Which is equivalent to
        \[
        (a+b)
        \]
        To express this in terms of nand, we first express it in terms
        of not, and. The formula is equivalent to
        \[
        \ov{\ov{a} \ov{b}}
        \]
        In other words,
        \[
        \overline{a} \vert \overline{b}
        \]
        or
        \[
        \left( a \vert a \right) \vert \left(b \vert b \right)
        \]
    \item
        We know that $\overline{a}$ is equivalent to $a \downarrow a$. As in the
        previous case, we can work with the simplified form $a+b$. We know
        that this is equivalent to
        \[
        \overline{ a \downarrow b}
        \]
        by the definition of $\downarrow$.
        Therefore,  $a + b $ is equivalent to
        \[
        \left( a \downarrow b \right) \downarrow \left( a \downarrow b \right)
        \]
    \end{enumerate}
\item
    Show how to use a truth table in order to construct a conjunctive
    normal form for any Boolean formula $W$. Hint: Consider the
    disjunctive normal form for $\neg W$.\\
    \noindent {\bf  ANSWER:}

    The method is completely dual to the method shown for getting the
    DNF  of a formula. Let's recall how we get a DNF from the truth
    table values of an expression $W$: We pick all rows where $W$
    is true. To get the DNF, we or together one term for each of those
    rows. The term is constructed from the values of the variables
    take on in that row. For example if in that row the variables
    have values $x = 0$, $y= 1$, $z=1$, then the term from that row
    will be $\overline{x} y z$.

    To construct a CNF, we simply do the opposite at every stage
    of the DNF construction: We pick all rows where $W$ is false.
    To get the CNF, we and together one term for each of those rows.
    The term is constructed from the values of the variables take on in
    that row. For example if in that row the variables have values
    $x= 0$ , $y= 1$ , $z =1$, then the term from that row will be
    $( x + \overline{y} + \overline{z})$.

\item
    Euclid proved that there are an infinite number of primes, by
    assuming that  $ n$ is the highest prime, and exhibiting a number
    that he proved must either be prime itself, or else have a prime
    factor greater than  $ n $. Write a scheme program to find
    the smallest  $ n$ for which Euclid's proof does
     {\em not} provide an actual prime number.\\
    \noindent {\bf  ANSWER:}
    There was some confusion as to whether the question wants
    you i) to use all primes less than a given prime $n$,
    {\em i.e.} to generate the primes in some other fashion
    an then apply Euclid's proof to an initial sequence of primes
    or ii) to use  {\em all primes generated by Euclid's proof}. These
    are not the same since Euclid's proof skips over many primes.
    For example, starting with $2$, Euclid's proof generates $3$, $7$ for
    the next two primes, skipping over $5$. The program we include here
    uses interpretation i).

\vskip 10pt

\beginlisp
(define (prime? x)    ; tests whether x is prime
 (define (iter count)
  (cond   ((> (* count count) x) true)
          ((= 0 (remainder x count)) false)
          (else (iter (inc count)))))
 (iter 2))
\endlisp

\vskip 10pt

\beginlisp
(define ones (cons-stream 1 ones))   ; 1,1,1,1,1,1,...
\endlisp

\vskip 10pt

\beginlisp
(define integers (cons-stream 1 (add-streams ones integers)))
;1,2,3,4,....
\endlisp

\vskip 10pt

\beginlisp
(define primes (stream-filter prime? integers))  ; 2,3,5,7,11,...
\endlisp

\vskip 10pt

\beginlisp
(define (mul-first-n-primes n) ; returns the product of the first
 (define (iter i answer)       ; n primes
  (cond ((= i 0) answer)
        (else (iter (dec i) (* (stream-ref primes i) answer)))))
 (iter n 1))
\endlisp

\vskip 10pt 
\beginlisp
; returns first composite number of the form p1 p2 ... pi + 1
\endlisp

\beginlisp
(define (find-first-composite)
 (define (iter i)
  (cond ((prime? (inc (mul-primes i)))
         (iter (inc i)))
        (else (display i)
              (display "   ")
              (inc (mul-first-n-primes i)))))
 (iter 2))
\endlisp

\vskip 10pt

\beginlisp
(find-first-composite)
6
;Value: 30031
\endlisp

\vskip 10pt

    \item
    You have proved before that a truth table with $n$ variables
    has $2^{n}$ rows.
    \begin{enumerate}
    \item
    How many different Boolean functions with $n$ variables are
    there? \\
    \noindent {\bf  ANSWER:}
    $2^{2^n}$
    \item For $n = 2$, list all the functions and identify as many as
    you can by name. \\
    \noindent {\bf  ANSWER:}
\[
    \begin{array}{c|c|c|c|c|c|c|c|c|c|c|c|c|c|c|c|c|c}
        a   &   b   &   0   &a \wedge b&\ov{a\ra b} &   a   &\ov{b\ra a} &   b  & a \oplus b &a \vee b &a\downarrow b& a \lra b& \ov{b}& b \ra a &\ov{a} & a\ra b&a \vert b&  1 \\\hline
        0   &   0   &   0   &   0      &   0          &   0   &   0         &   0  &   0   &   0   &   1      &   1    &   1   &    1    &   1   &   1   &   1   &  1 \\
        0   &   1   &   0   &   0      &   0          &   0   &   1         &   1  &   1   &   1   &   0      &   0    &   0   &    0    &   1   &   1   &   1   &  1 \\
        1   &   0   &   0   &   0      &   1          &   1   &   0         &   0  &   1   &   1   &   0      &   0    &   1   &    1    &   0   &   0   &   1   &  1 \\
        1   &   1   &   0   &   1      &   0          &   1   &   0         &   1  &   0   &   1   &   0      &   1    &   0   &    1    &   0   &   1   &   0   &  1
    \end{array}
    \]
    (Aside: Note that $\oplus$ is just the negation of $\lra$. )
    \end{enumerate}

\item
    Prove by induction that for $n > 5 $ , $2^{n} > n^{2}$. \\
    \noindent {\bf  ANSWER:}
    Actually, we can take $n \ge 5$.

    Base Case: $n = 5$. Then $2^5 = 32$, which is greater than
    $5^2$, which is $25$ \\

    Inductive step: Suppose for a fixed $n$ that $ 2^n > n^2 $
    We want to show that $2^{n+1} > (n+1)^2$.
    Multiplying the n-th case by $2$, we get
    \[
    2 \cdot 2^n > 2n^2,
    \]
    or
    \[
    2^{n+1} > 2n^2.
    \]
    So all we have to show is that
    \[
    2n^2 > (n+1)^2
    \]
    for all $n \ge 5$. Expanding out, this is true if and only if
    \[
    n^2 -2n -1 > 0
    \]
    for all $n \ge 5$. This in turn could be proven by induction
    or we could use the following trick. If we add 2 to the left hand side
    it becomes a perfect square:
    \[
    n^2 - 2n + 1 > 2
    \]
    In other words,
    \[
    (n+1)^2 > 2
    \]
    But this is true if and only if
    \[
    (n+1) > \sqrt{2}
    \]
    or equivalently
    \[
    n > \sqrt{2} -1 \approx 0.4
    \]
    So it is certainly true that
    \[
    2n^2  > (n+1)^2
    \]
    for all $n \ge 5$ and so
    \[
    2^{n+1} > 2n^2 > (n+1)^2
    \]
    and we have proved the inductive step. Q.E.D.
\item
    Guess the number of different ways for $n$ people to arrange
    themselves in a straight line, and prove your guess is correct
    by induction.\\

    \noindent {\bf  ANSWER:}  The number of ways for $n$ people to arrange themselves is $n!$.
    We will see why a bit later, but all we have to do here is prove
    the formula by induction.\\

    Base Case: The number of ways one person can arrange themselves
    in line is $1$, which agrees with $1!$ \\

    Inductive Step: Assume that for a given $n$ the number of ways $n$ people
    can arrange themselves in line is $n$. We would like to
    show that the number of ways $n+1$ people can arrange themselves in line
    is then necessarily $(n+1)!$. We can do this by breaking the
    process in to two steps. First we line up the first $n$ people
    and then we pick a spot for the last person to go to. By the
    rule of independent outcomes (which we assume you just sort of know
    about from common sense - we'll see it again when we do probability)
    the number of ways to do both things in sequence is the product
    of the number of ways to do each separate thing. Lining up $n$ people
    can be done in $n!$ ways by the induction hypothesis. The $(n+1)^{\text{st}}$
    has $n+1$ places to go - either before all the people, or after the
    first person, or after the second person \ldots. (Technically, we should
    also prove that there are $n+1$ places to go by another induction, but
    we'll skip this for our sanity and since the claim is sufficiently
    believable).  Therefore the number
    of ways to arrange $n+1$ people is the product of the two numbers,
    \[
    (n+1)\cdot n! = (n+1)!
    \]
    And so the inductive step isproven. Q.E.D.
\item
    Use logic with quantifiers and predicates to model the following
    three statements:
    \begin{itemize}
        \item All students are taking classes.
        \item Some students are not motivated.
        \item Some people taking classes are not motivated.
    \end{itemize}
    Prove, using resolution methods, that the third statement follows
    logically from the first two. (Reminder: You must take the conjunction
    of the first two statements and the negation of the third, and
    derive a contradiction.)\\
    \noindent {\bf  ANSWER:}
    \begin{enumerate}
    \item
        The universe is the set of all people. We need the following
        predicates
        \[
        S(x) \text{ if and only if $x$ is a student}
        \]
        \[
        M(x) \text{ if and only if $x$ is motivated}
        \]
        \[
        C(x) \text{ if and only if $x$ is taking classes}
        \]
    The sentences translate to
        \[
        \forall x S(x) \rightarrow C(x)
        \]
        \[
        \exists x S(x) \wedge \neg M(x)
        \]
        \[
        \exists x C(x) \wedge \neg M(x)
        \]
    \item
    Let's prove that the third follows from the first two.
    We'll go about it like the hint says, by assuming the
    first two and the negation of the third (we don't really have to
    take the conjunction of the first two statements, we'll just list them
    separately in our proof)
        \begin{align*}
            1. \quad & \forall x S(x) \rightarrow C(x) \hfill &\\
            2. \quad & \exists x S(x) \wedge \neg M(x)  \hfill &\\
            3. \quad & \neg \exists x C(x) \wedge \neg M(x)
                \hfill & \text{1.-3. are the assumptions we are making}\\
            4. \quad & \forall x  \neg C(x) \vee M(x)
                \hfill &\text{ by  3. and  the properties of quantifiers } \\
            5. \quad & S(p) \wedge \neg M(p)
               \hfill &\text{ for some $p$, by existential instantiation  of 2.} \\
            6. \quad & S(p) \rightarrow C(p)
               \hfill &\text{by universal instatiation of 1.} \\
            7. \quad & \neg S(p) \vee C(p)
                \hfill &\text{ by the definition of $\rightarrow$} \\
            8. \quad & \neg C(p) \vee M(p)
               \hfill &\text{ by universal instatiation of 4.} \\
            9. \quad & \neg S(p) \vee M(p)
                \hfill &\text{ by resolution of 7. and 8. } \\
           10. \quad & \neg \left( \neg S(p) \vee M(p) \right)
               \hfill &\text{ by 5. and Demorgan's Laws }
        \end{align*}
       This is a contradiction, since 9. and 10. are negations of each other.
        Therefore if 1. and 2. hold, 3. cannot hold. Q.E.D.
    \end{enumerate}

\item  The following algebraic idea is central for Karnaugh maps.
    Karnaugh maps are a method of minimizing the size of circuits for
    digital logic design.
    \begin{enumerate}
    \item
        Using algebraic manipulation, prove that the two Boolean formulae
        below are equivalent. (Hint: $x(a+ \overline{a})$ is equivalent
        to $x$.)
        \[
        x\overline{y} + y\overline{z} +  \overline{x}z \qquad \textrm{and} \qquad
         \overline{x}y +  \overline{y}z + x\overline{z}
        \]
    \noindent {\bf  ANSWER:}
    We'll show that the two formulas have the same canonical disjunctive normal form (CDNF).
    The formulas are already in disjunctive normal form. The CDNF is the DNF
    you would get using a truth table. What distinguishes it from DNF is that each
    variable in the formula must appear in each of the terms being or'ed togther.
    The formulas above do not include all of the variables $x$, $y$, $z$ in each term.
    We can fix by using the trick suggested by the hint. The first formula expands to
    \begin{align*}
        x\overline{y} + y\overline{z} +  \overline{x}z
            & \Leftrightarrow  x \overline{y}(z + \overline{z})
                + (x + \overline{x}) y \overline{z}
                + \overline{x}(y + \overline{y})z \\
            & \Leftrightarrow x \overline{y} z +  x \overline{y} \overline{z}
                + x y \overline{z} + \overline{x} y \overline{z}
                + \overline{x} y z + \overline{x} \overline{y} z
    \end{align*}
    The second formula expands to
    \begin{align*}
         \ov{x}y +  \ov{y}z + x\ov{z}
            & \Leftrightarrow \ov{x}y (z  +  \ov{z})
                + ( x + \ov{x}) \ov{y}z + x(y + \ov{y})\overline{z}\\
            & \Leftrightarrow \ov{x}y z + \ov{x} y \ov{z} + x \ov{y} z + \ov{x} \ov{y} z
             + x y \ov{z} + x \ov{y} \ov{z}
    \end{align*}
    Now the two expanded formulas are the same up to permutation, so the original formulas are equivalent.
    \item Verify your results using a truth table.\\
    \noindent {\bf  ANSWER:}
     \[
    \begin{array}{c|c|c|c|c}
        x   &   y   &   z   &   x\ov{y} + y\ov{z} +  \ov{x}z  & \ov{x}y +  \ov{y}z + x\ov{z} \\\hline
        0   &   0   &   0   &   0   &  0 \\
        0   &   0   &   1   &   1   &  1 \\
        0   &   1   &   0   &   1   &  1 \\
        1   &   0   &   0   &   1   &  1 \\
        0   &   1   &   1   &   1   &  1 \\
        1   &   0   &   1   &   1   &  1 \\
        1   &   1   &   0   &   1   &  1 \\
        1   &   1   &   0   &   0   &  0
    \end{array}
    \]
    Yes!
    \end{enumerate}
\item
    The exclusive-or operator $\oplus$, is defined by the rule that $a \oplus
    b$
    is true whenever $a$ or $b$ is true but not both.
    \begin{enumerate}
    \item Calculate $x \oplus x$ , $x \oplus \neg x$ ,
    $x \oplus 1$ , $x \oplus 0$.\\
    \noindent {\bf  ANSWER:}
    $0$, $1$, $\neg x$ , $x$
    \item Prove or disprove that $x + y \oplus z = (x+ y) \oplus (x+ z)$\\
    \noindent {\bf  ANSWER:}
    This is not true. If we let $x$ be true, the left-hand side will always
    be true, whereas the right-hand side will always be false.
    \item Prove or disprove that $x  \oplus (y +  z) = ( x  \oplus y) + ( x  \oplus z)$\\
    \noindent {\bf  ANSWER:}
    This is not true. If we let $x$ be true, the left-hand side is equivalent
    to $\neg (y + z)$, {\em i.e.} nor of $y,z$,  and the right hand is equivalent
    to $(\neg y ) \vee (\neg z)$, {\i.e} nand of $y,z$. And these are definitely
    not equivalent.
    \item
    Write conjunctive normal form and disjunctive normal form formulae
    for  $x  \oplus  y$\\
    \noindent {\bf  ANSWER:}
    \begin{align*}
    & \text{CNF:} (a + b)( \ov{a} + \ov{b} ) \\
    & \text{DNF:} \ov{a} b+ a \ov{b}
    \end{align*}
    \item The exclusive-or operator is not  {\em complete.} Which of the
    three operators \{and, or, not\} can be combined with exclusive-or
    to make a  {\em complete} set.\\
    \noindent {\bf  ANSWER:}
    \begin{itemize}
        \item $\oplus$ and $\neg$ are not complete. This is not the
        most straightforward thing in the world to show. If they
        were complete, we would be able to generate any four-bit
        sequence (column of a truth table) with the operators
        just starting from $0011$, $0101$. The problem is that
        both of the starting sequences have an even number of
        ones, and $\oplus$, $\neg$ will keep the number of ones
        even, so we will never be able to generate $1000$, for
        example.
        \item $\oplus$ and $\vee$ are not complete. If they
        were complete, we would be able to generate any four-bit
        sequence (column of a truth table) with the operators
        just starting from $0011$, $0101$. However both $\oplus$
        and $\vee$ cannot change the $0$ in the first positions of
        $0011$, $0101$ to a
        $1$, so we will never be able to generate any sequence
        starting with a $1$.
        \item $\oplus$ and $\wedge$ are not complete for the same
        reason as the previous case.
    \end{itemize}
    \end{enumerate}
\item
    The  $n^{\mathrm{th}}$ triangle number  $T_{n}$ is defined to be the sum
    of the first  $n$ integers. \\
    \begin{enumerate}
    \item
    Prove by induction that  $T_{n} =  n( n + 1)/2.$ \\
    \noindent {\bf  ANSWER:} \\
    Base case: $T_1 = 1 = 1(1 + 1)/2$ \\
    Inductive Step: Suppose that for a fixed $n$, it is true that  $T_n = \frac{n(n+1)}{2}$.
    We need to show that it follows necessarily that $T_{n+1} = \frac{(n+1)(n+2)}{2}$.
    We can do this by expressing $T_{n+1}$ in terms of $T_{n}$ and manipulating the result
    \begin{align*}
        T_{n+1} &= T_n + (n+1) = \frac{n(n+1)}{2} + (n+1) \\
                &= (n+1)(\frac{n}{2} + 1) = \frac{(n+1)(n+2)}{2}.
    \end{align*}
    Therefore the inductive step holds and it follows
    that $T(n) = \frac{n(n+1)}{2}$ for every $n$.
    \item Prove algebraically using (a), that
    \[
        n^{3} + (1+2+\cdots +(n - 1))^{2}
            = (1+2+\cdots (n-1) +  n)^{2}
    \]
    \noindent {\bf  ANSWER:}
    We want to show that
    \begin{align*}
        n^3 &=(1+2+\cdots +n)^{2} - (1+2+\cdots +(n - 1))^{2} \\
            & = T_n^2 - T_{n-1}^2
    \end{align*}
    Using the formula for $T_n$, the right hand-side becomes
    \begin{align*}
       \frac{n^2(n+1)^2}{4} - \frac{(n-1)^2n^2}{4} &= \frac{n^2}{4} ( (n+1)^2 - (n-1)^2) \\
                                                   &= \frac{n^2}{4} ( 4n) \\
                                                   &= n^3
    \end{align*}
    Q.E.D.
    \item
    Using (b) guess a formula for
    \[
    C_n = 1^{3} + 2^{3} + 3^{3} + \cdots  +  n^{3},
    \]
    and prove it by induction.\\
    \noindent {\bf  ANSWER:}
    This our first encounter with  a telescoping series ({\em c.f. } problem~set~2)
   \begin{align*}
        1^{3} + 2^{3} + 3^{3} + \cdots  +  n^{3} & =
          (T_1^2 - T_0^2) + (T_2^2 -T_1^2)  + \cdots  + (T_n^2 - T_{n-1}^2) \\
            &= ( -T_0^2 + T_n^2 )\\
         &= T_n^2 = \frac{n^2(n+1)^2}{4}
  \end{align*}
    Even though it is fairly straightforward from the above, we'll prove that
    \[
     C_n = 1^{3} + 2^{3} + 3^{3} + \cdots  +  n^{3} = \frac{n^2(n+1)^2}{4}
    \]
    by induction, just for practice. \\
    Base case: $C_1 = 1^3 =1$, and  $\frac{1^2 2^2}{4} = 1$ \\
    Inductive Step: Suppose that for a fixed $n$, that $C_n = \frac{n^2(n+1)^2}{4}$.
    We want to show that this necessarily implies $C_{n+1} = \frac{(n+1)^2(n+2)^2}{4}$. We start
    by expressing $C_{n+1}$ in terms of $C_n$ and then manipulate the result:
    \begin{align*}
    C_{n+1} & = C_n + (n+1)^3\\
            & = \frac{n^2(n+1)^2}{4} + (n+1)^3 \\
            & = (n+1)^2 (\frac{n^2}{4} + (n+1) ) \\
            & = (n+1)^2  \frac{n^2 + 4n +4}{4} \\
            & = (n+1)^2 \frac{(n+2)^2}{4}.
    \end{align*}
    This is what we wanted to show, so the inductive step is proven.\\
    Q.E.D.
    \end{enumerate}
\item Guess a formula for the sum below, and prove you are right by
    induction.
    \[
    1 + 1(2) + 2(3) + 3(4) + \cdots  +  n(n + 1)
    \]
    \noindent {\bf  ANSWER:}
    The $1$ at the beginning of the formula doesn't fit the pattern of the
    tail of the sum, so we'll figure a formula for
    \[
    A_{n} := 1(2) + 2(3) + 3(4) + \cdots  +  n(n + 1)
    \]
    and then we can add 1 to it at the end. We can reduce this sum
    into two more basic sums:
    \begin{align*}
    1(2) + 2(3) + 3(4) + \cdots  +  n(n + 1)
        &= 1(1 + 1) + 2(2 + 1) + \cdots + n(n+1) \\
        &= (1^2 + 2^2 + \cdots n^2) + (1 + 2 +  \cdots n)
    \end{align*}
    The second sum is one we already have a closed form for. I claim
    that the formula for the first sum is
    \[
    (1^2 + 2^2 + \cdots n^2) = \frac{n(n+1)(2n+1)}{6},
    \]
    so the original sum is
    \begin{align*}
    A_n := 1 + 1(2) + 2(3) + 3(4) + \cdots  +  n(n + 1)
        &= 1 + \frac{n(n+1)(2n+1)}{6} + \frac{n(n+1)}{2} \\
        &= 1 + \frac{n(n+1)}{2}(\frac{2n+1}{3} +1) \\
        &= 1 + \frac{n(n+1)(2n+4)}{6}.
    \end{align*}
    The beauty of proof-by-induction is you don't have to know at all how
    I came up with the formula for the sum. You can still use
    induction to prove that it is correct. Let's do that now: \\
    Base Case: $A_0 = 1$,  $1 + \frac{0(0+1)(2 \cdot 0+4)}{6} = 1$ \\
    Inductive Step: Suppose that for a fixed $n$, $A_n = 1 + \frac{n(n+1)(2n+4)}{6}$
    we want to show that it follows necessarily that $A_{n+1} = 1 + \frac{(n+1)(n+2)(2n+6)}{6}$. We start
    by expressing $A_{n+1}$ in terms of $A_n$ and then manipulate the result:
    \begin{align*}
    A_{n+1} & = A_n + (n+1)(n+2) \\
            & = 1 + \frac{n(n+1)(2n+4)}{6} + (n+1)(n+2) \\
            & = 1 + (n+1) (\frac{n(2n+4)}{6} + (n+2)) \\
            & = 1 + (n+1) (\frac{ 2n^2 + 10n + 12}{6})  \\
            & = 1 + (n+1) \frac{(n+2)(2n+6)}{6}.
    \end{align*}
    This is what we wanted to show, so the inductive step is proven.\\
    Q.E.D.
\end{enumerate}

\end{document}
