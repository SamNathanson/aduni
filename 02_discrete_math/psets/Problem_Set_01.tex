%&LaTeX
\documentclass{article}

\newcommand{\tab}{\makebox[4em]{}}


\begin{document}


\begin{center}
\textbf{ArsDigita University}


\end{center}

\begin{center}
\textbf{Month 2: Discrete Mathematics - Professor Shai Simonson}


\end{center}

\begin{center}
\textbf{Problem Set 1 -- Logic, Proofs, and Mathematical Reasoning}


\end{center}

\begin{enumerate}
\item { Logic Proofs }
    \begin{enumerate}
    \item   Prove that $a \rightarrow b$
            is equivalent to $\neg b
            \rightarrow \neg a$
            using a truth table. \\
    \item   Prove it using {\em algebraic} identities. \\
    \item   Prove that $a \rightarrow b$ is not
            equivalent to $ b \rightarrow a$.
    \end{enumerate}
\item {Aristotle's Proof that the Square Root of Two is Irrational.}\\
    \begin{enumerate}
    \item Prove the  {\em lemma,} used by Aristotle in his proof,
        which says that if $n^{2}$ is even, so is
        $n$. (Hint: Remember that $a \rightarrow  b$
        is equivalent to $b \rightarrow \neg a$.\\
    \item
        Prove that the square root of 3 is irrational using Aristotle's
        techniques. Make sure to prove the appropriate lemma.\\
    \item
        If we use Aristotle's technique to  {\em prove} the untrue assertion
        that the square root of 4 is irrational, where  {\em exactly} is
        the hole in the proof?\\
    \item Using the fact that the square root of two is irrational, prove
        that $\sin (\frac{\pi}{4})$ is irrational.
    \end{enumerate}
\item
    In ADU-ball, you can score 11 points for a goal, and 7 for a
    near miss. \\
    \begin{enumerate}
    \item
    Write a Scheme program that prints out the number of goals and
    the number of near misses to achieve a given total greater than
    60.\\
    \item Prove that you can achieve any score greater than 60. Think
    inductively and experiment.
    \end{enumerate}
\item Prove by induction that there are 2$^{\mathit{n}}$ possible
      rows in a truth table with  $ n$ variables.
\item   In the restroom of a fancy Italian restaurant in Mansfield,
        MA, there is a sign which reads:  {\em Please do not leave
        valuables or laptop computers in your car.}
\item   Assuming that a laptop computer is considered a valuable, prove
        using formal logic, that the sentence  {\em Please do not leave
        valuables in your car} is equivalent to the sign in the restroom.
        Prove that  {\em Please do not leave laptops in your car} is not
        equivalent.
\item   Prove that $a | b$, $a$
        nand $b$, which is defined to be
    $\neg (a \wedge b)$, is
    complete. Write
    $( a \rightarrow b) \rightarrow b$
    using just $|$ (nand), then using just
    $\downarrow$ (nor).

\item
    Show how to use a truth table in order to construct a conjunctive
    normal form for any Boolean formula $W$. Hint: Consider the
    disjunctive normal form for $\neg W$.


\item
    Euclid proved that there are an infinite number of primes, by
    assuming that  $ n$ is the highest prime, and exhibiting a number
    that he proved must either be prime itself, or else have a prime
    factor greater than  $ n $. Write a scheme program to find
    the smallest  $ n$ for which Euclid's proof does
     {\em not} provide an actual prime number.

\item
    You have proved before that a truth table with $n$ variables
    has $2^{n}$ rows.
    \begin{enumerate}
    \item
    How many different Boolean functions with $n$ variables are
    there? \\
    \item For $n = 2$, list all the functions and identify as many as
    you can by name.
    \end{enumerate}

\item
    Prove by induction that for $n > 5 $ , $2^{n} > n^{2}$.


\item
    Guess the number of different ways for $n$ people to arrange
    themselves in a straight line, and prove your guess is correct
    by induction.


\item
    Use logic with quantifiers and predicates to model the following
    three statements:
    \begin{itemize}
    \item All students are taking classes.
    \item Some students are not motivated.
    \item Some people taking classes are not motivated.
    \end{itemize}
    Prove, using resolution methods, that the third statement follows
    logically from the first two. (Reminder: You must take the conjunction
    of the first two statements and the negation of the third, and
    derive a contradiction.)

\item    The following algebraic idea is central for Karnaugh maps.
    Karnaugh maps are a method of minimizing the size of circuits for
    digital logic design.

    \begin{enumerate}
    \item
    Using algebraic manipulation, prove that the two Boolean formulae
    below are equivalent. (Hint: $x(a+ \neg a)$ is equivalent
    to $x$.)
    \[
    \neg yx + \neg zy + \neg xz \qquad \textrm{and} \qquad
    \neg xy + \neg yz + \neg zx
    \]
    \item Verify your results using a truth table.
    \end{enumerate}

\item
    The exclusive-or operator $\oplus$, is defined by the rule that $a \oplus
    b$
    is true whenever $a$ or $b$ is true but not both.
    \begin{enumerate}
    \item Calculate $x \oplus x$ , $x \oplus \neg x$ ,
    $x \oplus 1$ , $x \oplus 0$.\\
    \item Prove that $x + y \oplus z = (x+ y) \oplus (x+ z)$\\
    \item Prove that $x  \oplus (y +  z) = ( x  \oplus y) + ( x  \oplus z)$\\
    \item
    Write conjunctive normal form and disjunctive normal form formulae
    for  $x  \oplus  y$\\
    \item The exclusive-or operator is not  {\em complete.} Which of the
    three operators \{and, or, not\} can be combined with exclusive-or
    to make a  {\em complete} set.
    \end{enumerate}

\item
    The  n th triangle number  $T_{n}$ is defined to be the sum
    of the first  $n$ integers. \\
    \begin{enumerate}
    \item
    Prove by induction that  $T_{n} =  n( n + 1)/2.$ \\
    \item Prove algebraically using (a), that  $n^{3} + (1+2+\cdots +(n - 1))^{2}
        = (1+2+\cdots (n-1) +  n)^{2}$\\
    \item
    Using (b) guess a formula for
    \[
    1^{3} + 2^{3} + 3^{3} + \cdots  +  n^{3},
    \]
    and prove it by induction.
    \end{enumerate}

\item
    Guess a formula for the sum below, and prove you are right by
    induction.
    \[
    1 + 1(2) + 2(3) + 3(4) + \cdots  +  n(n + 1)
    \]

\end{enumerate}

\end{document}
