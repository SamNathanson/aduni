 %&LaTeX

\documentclass{article}
\usepackage{fullpage}
\usepackage{amsmath}
\newcommand{\fr}[2]{\frac{#1}{#2}}

\begin{document}


\begin{center}
\textbf{ArsDigita University}

\textbf{Month 2: Discrete Mathematics - Professor Shai Simonson}

\textbf{Problem Set 4 -- Induction and Recurrence Equations}

\textbf{Thanks to Jeffrey Radcliffe and Joe Rizzo for many of the
solutions.\\
Pasted together and modified by Dimitri Kountourogiannis}

\end{center}

\begin{enumerate}
\item
What's wrong with the following proofs by induction?

\begin{enumerate}
\item
All binary strings are identical. The proof is by induction on the
size of the string. For \textit{n=}0 all binary strings are empty
and therefore identical. Let \textit{X =
b}$_{\mathit{n}}$\textit{b}$_{\mathit{n-1}}$\textit{\dots
b}$_{\mathit{1}}$\textit{b}$_{\mathit{0}}$ be an arbitrary binary
string of length \textit{n+}1. Let \textit{Y =
b}$_{\mathit{n}}$\textit{b}$_{\mathit{n-1}}$\textit{\dots
b}$_{\mathit{1}}$ and \textit{Z =}
\textit{b}$_{\mathit{n-1}}$\textit{\dots
b}$_{\mathit{1}}$\textit{b}$_{\mathit{0}}$. Since both \textit{Y}
and \textit{Z} are strings of length less than \textit{n+}1, by
induction they are identical. Since the two strings overlap,
\textit{X} must also be identical to each of them. \\
\textbf{ANSWER} \\
 This is not a subtly flawed proof. It is a terribly flawed proof.
 The problem is with the induction step. It is never true that the
 pasting together of two identical strings is identical to the
 strings. Furthermore, the $n=1$ case fails for another reason:
 A string of length 1 is not the pasting together of two strings
 of length 0.
\\

\item
Any amount of change greater than or equal to twenty can be gotten
with a combination of five cent and seven cent coins. The proof is
by induction on the amount of change. For twenty cents use four
five-cent coins. Let \textit{n} \texttt{>} 20 be the amount of
change. Assume that \textit{n} = 7\textit{x +} 5\textit{y} for
some non-negative integers \textit{x} and \textit{y.} For any
\textit{n \texttt{>}} 20, either \textit{x \texttt{>}} 1, or
\textit{y \texttt{>}} 3. If \textit{x \texttt{>}} 1, then since
3(5) -- 2(7) = 1, \textit{n+}1 = 5(\textit{y}+3)+7(\textit{x}--2).
If \textit{y \texttt{>}} 3, then since 3(7) -- 4(5) = 1,
\textit{n+}1 = 7(\textit{x}+3)+5(\textit{y}--4). In either case,
we showed that \textit{n+}1 = 7\textit{u +} 5\textit{v} where
\textit{u} and \textit{v} are non-negative integers. \\
\textbf{ANSWER} \\
For any  $n > 20$, either  $x > 1$  or $y > 3$
is false. For example, if $n=21$, then $21 = 0 \cdot 5 + 3 \cdot
7$.
\end{enumerate}

\item
Prove by induction that:


\begin{enumerate}
\item
The \textit{n}th Fibonacci number, $F_n$  equals
\[
G_n = \fr{1}{\sqrt{5}}\left[(\fr{1 +\sqrt{5}}{2})^{n} - (\fr{1 +
\sqrt{5} }{2})^n
 \right] \]
 where \textit{F}$_{\mathit{0}}$ = 0
and \textit{F}$_{\mathit{1}}$ = 1.
\\ \textbf{ANSWER}
\\
    Base Case:
These hold since
\[
G_0 = \fr{1}{\sqrt{5}}\left[(\fr{1 +\sqrt{5}}{2})^{0} - (\fr{1 +
\sqrt{5} }{2})^0 \right] = \fr{1}{\sqrt{5}}(1-1) = 0
\]
and
\[
G_1 = \fr{1}{\sqrt{5}}\left[(\fr{1 +\sqrt{5}}{2})^{1} - (\fr{1 +
\sqrt{5} }{2})^1 \right] = \fr{1}{\sqrt{5}} \left[
\fr{1+\sqrt{5}}{2} - \fr{1- \sqrt{5}}{2} \right] = 1
\]
Induction Step: Suppose that $G_n = F_n$ and $G_{n-1} = F_{n-1}$
We would like to show
$$ G_{n+1} = F_{n+1}$$
    Let $a = (\fr{1}{2} + \fr{\sqrt{5}}{2})$\\
    Let $b = (\fr{1}{2} - \fr{\sqrt{5}}{2})$\\
We will use the fact that $a$, $b$ are roots of  $r^2 - r - 1 = 0$
and hence
\[
a^2 = a + 1 \qquad b^2 = b + 1
\]
Starting with $G_{n+1}$,
\begin{eqnarray*}
G_{n+1} = \fr{1}{\sqrt{5}}(a^{n+1} - b^{n+1}) & = &
\fr{1}{\sqrt{5}} ( a^2 a^{n-1} - b^2 b^{n-1} ) \\
& = & \fr{1}{\sqrt{5}} ( (a+1) a^{n-1} - (b+1) b^{n-1} ) \\
& = & \fr{1}{\sqrt{5}}(a^n - b^n) + \fr{1}{\sqrt{5}}
                (a^{n-1} - b^{n-1})) \\
& = & F_n + F_{n-1}\\
& = & F_{n+1}
\end{eqnarray*}
So the induction step is proven. QED.

\item
The sum of the geometric series 1 + \textit{a} + \textit{a}$^{2}$
+ \dots  + \textit{a}$^{n}$ equals
(1--\textit{a}$^{\mathit{n}}$$^{+1}$)/(1--\textit{a}), where
\textit{a} does not equal
one.\\
\vskip 10pt  \textbf{ANSWER}

    Base Case: $n = 0 $\\
    $$ \fr{(1 - a^{0+1})}{1 - a} = 1 $$

    Inductive Step:
    Suppose
    \[
     1 + a + a^2 + \ldots + a^n =
    \fr{1-a^{n+1}}{1-a}
    \]
    Then
    \begin{eqnarray*}
    1 + a + a^2 + \ldots + a^n + a^{n+1} & = &
    \fr{1-a^{n+1}}{1-a} +  a^{n+1}\\
    & = & \fr{1 - a^{n+1} + a^{n+1} - a^{n+2}}{1 - a} \\
    & = & \fr{1- a^{n+2}}{1 -a}
    \end{eqnarray*}
    And the inductive step is proven. Q.E.D.
\item
21 divides 4$^{\mathit{n+}}$$^{1}$ + 5 $^{\mathit{2n-}}$$^{1}$\\
\vskip 10pt  \textbf{ANSWER}

  Base Case:  If $n = 1$, then
     $$4^{2} + 5^{1} = 21,$$ which is divisible by $21$.
  Inductive step: Assume that for some $n$
  \[
  4^{n+1} + 5^{2n-1}
  \]
  is divisible by 21.
  Then
  \begin{eqnarray*}
  4^{n+2} + 5^{2n+1} & = & 4\cdot 4^{n+1} + 5^2 \cdot 5^{2n-1} \\
                     & = & 4 \cdot 4^{n+1} + 4 \cdot 5^{2n-1} + 21
                     \cdot 5^{2n-1} \\
                     & = & 4 ( \cdot 4^{n+1} + \cdot 5^{2n-1} ) + 21
                     \cdot 5^{2n-1}
  \end{eqnarray*}
  The first term is divisible by 21 by the induction hypothesis,
  and the second term is manifestly divisible by 21. Hence the
  induction step holds. Q.E.D.


\item The number of leaves in a
complete binary tree is one more than the number of internal
nodes. (Hint: Split the tree up into
two smaller trees).\\
\vskip 10pt  \textbf{ANSWER} \\
We induct on the depth of the tree. \\
Base Case: For a tree of depth $1$, there is one internal node and
two leaves. \\
Inductive step: Suppose for all complete binary trees $T$ of depth
at most $n$, that
\[
L(T) = I(T) + 1
\]
Where $L(t)$ is the number of leaves of $T$ and $I(T)$ is the
number of internal nodes of $T$.\\
Let $B_{n+1}$ be a complete binary tree of depth $n+1$. We would
like to verify
\[
L(B_{n+1}) \quad ?=? \quad I(B_{n+1}) + 1
\]
This will follow from the following relations between $B_{n+1}$
and its left and right subtrees $B_n$, $B_n'$.
\begin{itemize}
\item $I(B_{n+1}) = I(B_n) + I(B_n') + 1 \qquad \textrm{ (+1 for the
new root) }$
\item $ L(B_{n+1}) = L(B_n) + L(B'_n) $
\end{itemize}
So
\begin{align*}
    L(B_{n+1}) & = L(B_n) + L(B_n') \\
               & = (I(B_n) + 1) + (I(B_n') + 1) \qquad \textrm{by
               induction} \\
               & = (I(B_n) + I(B_n') + 1) \\
               & = I(B_{n+1}) + 1
\end{align*}

And the induction step is proven. Q.E.D.
\item A graph's edges can be covered
by \textit{n} edge-disjoint paths, but not \textit{n-1}, if and
only if the graph has \textit{n} pairs of odd-degree vertices.
(Euler discussed the case for \textit{n = 1).} \\\vskip 10pt
\textbf{ANSWER} \\
There are two directions to the proof.\\
 $\Leftarrow$ \\
Thanks to Sam Klein for the argument in the first direction. \\
Suppose that a graph $G$ has $n$ pairs of odd degree vertices,
$a_1, b_1, a_2, b_2 , ... , a_n, b_n$. Construct a graph $G'$ that
is $G$ with the $n$ edges $\{ a_1, b_1 \}, \cdots \{ a_n, b_n \}$
thrown in. In $G'$ all the vertices are of even degree, so there
is an Euler circuit, i.e. a path which traverses every edge of
$G'$ without going over any edge twice. Removing the edges we
added to $G$, we chop the loop into $n$ edge-disjoint paths that
cover $G$. \\

To prove the rest, we will use the following fact: \textit{ if $k$
edge-disjoint paths cover a graph $X$, then $X$ has at most $k$
pairs of odd-degree vertices}. \\

This is because an odd degree vertex must lie at the beginning or
the end of some path, or else there will be an outgoing edge for
each incoming edge at the vertex and so it will necessarily be
even degree.  So it follows that there cannot be $n-1$
edge-disjoint paths if there are $n$ pairs of odd-degree
vertices.\\

$\Rightarrow$ \\
 Conversely, if there are $n$ edge disjoint paths but
not $n-1$ (or any smaller number). We know then that there are at
most $n$ pairs of odd vertices. If there were fewer than $n$ pairs
of odd vertices then by the other direction of the proof, we would
have fewer than $n$ edge-disjoint paths that cover the graph,
which would contradict our assumption. Therefore there must be $n$
pairs of odd-degree vertices.

\end{enumerate}

\item
 Solve the following recurrence equations using the techniques
for linear recurrence relations with constant coefficients. State
whether or not each recurrence is homogeneous.\\
\begin{enumerate}
\item
\textit{a}$_{\mathit{n}}$ = 6\textit{a}$_{\mathit{n-1}}$
\ensuremath{-} 8\textit{a}$_{\mathit{n-2}}$, and
\textit{a}$_{\mathit{0}}$
= 4, \textit{a}$_{\mathit{1}}$ = 10.\\
\vskip 10pt  \textbf{ANSWER}
$a_n = 6a_{n-1} - 8a_{n-2}$, and $a_0 = 4, a_1 = 10$\\
This is a homogeneous recurrence equation.          \\
\begin{eqnarray*}
\frac{r^k} {r^{k-2}} &=& \frac{6r^{k-1} - 8k^{k-2}} {r^{k-2}}   \\
r^2 & = & 6r - 8                        \\
r^2 -6r + 8 & = & 0                     \\
(r - 2)(r - 4)  & = & 0                     \\
\end{eqnarray*}
The roots are $2$, $4$.
\begin{eqnarray*}
a_n & = & \alpha_1 2^n + \alpha_2 4^n               \\
a_0  & = & \alpha_1 2^0 +  \alpha_2 4^0 = \alpha_1 + \alpha_2 = 4 \\
a_1  & = & \alpha_1 2^1 + \alpha_2 4^1 = 2 \alpha_1 + 4 \alpha_2 = 10 \\
\end{eqnarray*}
$\alpha_1  = 3$ and $\alpha_2 = 1$. The solution is $a_n = 3(2^n)
+ 1(4^n)$.
\item
\textit{a}$_{\mathit{n}}$ = \textit{a}$_{\mathit{n-1}}$ +
2\textit{a}$_{\mathit{n-2}}$, and \textit{a}$_{\mathit{0}}$
= 0, \textit{a}$_{\mathit{1}}$ = 1.\\
\vskip 10pt  \textbf{ANSWER} This is a homogeneous recurrence
equation.
\\The solution is $a_n = \frac{1}{3}(2^n) - \frac{1}{3}(-1^n)$.

\item
\textit{a}$_{\mathit{n}}$ = 7\textit{a}$_{\mathit{n-1}}$
\ensuremath{-}10\textit{a}$_{\mathit{n-2}}$ +
3$^{\mathit{n}}$, and \textit{a}$_{\mathit{0}}$ = 0, \textit{a}$_{\mathit{1}}$ = 1.\\
\vskip 10pt  \textbf{ANSWER} This is a nonhomogenous equation.
First, determine the roots of the associated homogeneous equation:
\begin{eqnarray*}
\frac{r^k} {r^{k-2}} &=& \frac{7r^{k-1} - 10k^{k-2}} {r^{k-2}}  \\
r^2 & =&  7r - 10                       \\
r^2 - 7 + 10 & = & 0                        \\
(r - 5)(r - 2) & = & 0                      \\
\end{eqnarray*}
The roots are 5 and 2. Next, find the particular solution, setting
$p3^n = a_n$:
\begin{eqnarray*}
p3^n &=& 7p3^{n-1} - 10p3^{n-2} + 3^n               \\
\frac{p3^n}{3^n} & =& \frac{\frac{7}{3}p3^n - \frac{10}{9}p3^n + 3^n}{3^n}  \\
p & = & \frac{7}{3}p - \frac{10}{9}p + 1                    \\
\frac{9}{9}p - \frac{21}{9}p + \frac{10}{9}p & = & 1                \\
\frac{2}{9}p & = & - 1                              \\
p & = & -\frac{9}{2}                                \\
\end{eqnarray*}
\begin{eqnarray*}
a_n & = & \alpha_1 2^n + \alpha_2 4^n               \\
a_0  = 0 & = & \alpha_1 2^0 +  \alpha_2 4^0 = \alpha_1 + \alpha_2 - \frac{9}{2}\\
a_1  = 1 & = & \alpha_1 5^1 + \alpha_2 2^1 + \frac{9}{2}(3^1)
= 5 \alpha_1 + 2 \alpha_2 - \frac{27}{2}\\
\end{eqnarray*}
Solving these equations, $\alpha_1 = \frac{11}{6}$, and $\alpha_2
= \frac{16}{6}$. The final solution is
$$
a_n = \frac{11(5^n) + 16(2^n) - 27(3^n)}{6}.
$$
\item
\textit{a}$_{\mathit{n}}$ = 3 \ensuremath{-}
6\textit{a}$_{\mathit{n-1}}$ \ensuremath{-}
9\textit{a}$_{\mathit{n-2}}$, and \textit{a}$_{\mathit{0}}$ = 0,
\textit{a}$_{\mathit{1}}$ = 1. \\
\vskip 10pt  \textbf{ANSWER}This is not a homogeneous recurrence
equation.
\\The solution is $a_n = -\frac{3}{16}(-3^n) - \frac{1}{12}n(-3^n) + \frac{3}{16}$.
\end{enumerate}

\item
 A particular graph-matching algorithm on \textit{n} nodes,
works by doing \textit{n}$^{\mathit{2}}$ steps, and then solving a
new matching problem on a graph with one vertex less.

\begin{enumerate}
\item
 Show that the number of steps it takes to run the algorithm on a
graph with \textit{n} nodes is equal to the sum of the first
\textit{n}
perfect squares.\\
\vskip 10pt  \textbf{ANSWER}
 Taking $T_0 = 0$ as the base case, we can use substitution to show that this algorithm
is equal to the sum of the squares from one to $n$.
\begin{eqnarray*}
T_n & = & T_{n-1} + n^2 \\
& = & (T_n-1 + (n-1)^2) + n^2\\
& = & (T_n-2 + (n-2)^2) + (n-1)^2 + n^2\\
& = & T_n-r + (n-r)^2 + (n-(r+1))^2 + (n-(r+2))^2) \ldots (n-1)^2 + n^2\\
& = & 0 + 0 + 1^2 + 2^2 + 3^2 + \ldots + (n-1)^2 + n^2\\
& = & 1^2 + 2^2 + 3^2 + \ldots + (n-1)^2 + n^2\\
\end{eqnarray*}
\item
Derive the formula for the sum of the first \textit{n} perfect
squares by constructing an appropriate linear non-homogeneous
recurrence
equation and solving it.\\
\vskip 10pt  \textbf{ANSWER} The general equation for the
recurrence equation $a_n = a_{n-1} + n^2$ can be expressed as $a_n
= \alpha_1(1^n)
+ n(p_2 n^2 + p_1 n + p_0)(1^n)$. First solve to find $a^{(p)}_n$:\\\\
$a_n = n(p_2n^2 + p_1n + p_0) = p_2n^3 + p_1n^2 + p_0n \equiv (p_2(n-1)^3 + p_1(n-1)^2 + p_0(n-1)) + n^2$\\\\
$p_2n^3 + p_1n^2 + p_0n  \equiv( p_2n^3 - p_2 3 n^2  + p_2 3 n - p_2) + (p_1n^2 -p_1 2 n + p_1) + (p_0n - p_0) + n^2$\\
\begin{eqnarray*}
p_2 3 n^2  - p_2 3 n + p_2 + p_1 2 n - p_1 + p_0 - n^2 & = & 0\\
(3p_2 - 1)n^2 + (2p_1 - 3p_2)n + (p_2 - p_1 + p_0) & = & 0\\
(-1)n^2 + (0)n + (p_2 - p_1 + p_0) & = & 0\\
\end{eqnarray*}
Solving the series of equations, $p_2 = \frac{1}{3}$, $p_1 =
\frac{1}{2}$, and $p_0 = \frac{1}{6}$. So, $a^{(p)}_n =$
$$
n\left( \frac{n^2}{3} + \frac{n}{2} + \frac{1}{6} \right)
 = n \left( \frac{2n^2 + 3n + 1}{6} \right) = \frac{n(2n + 1)(n + 1)}{6}.
$$
Solving for $a_0 = 0$, $a^{(h)}_n = 0$. The closed form is then
$\frac{n(2n + 1)(n + 1)}{6}$.
\\\\
\item
Show that the time complexity of this algorithm is
\ensuremath{\Theta}\textit{(n}$^{\mathit{3}}$\textit{).} \vskip
10pt  \textbf{ANSWER} We can multiply the the solution from 4b to
equal $\frac{1}{6}(2n^3 + 3n^2 + n)$, and show that this function
 is both $O(n^3)$ and $\Omega(n^3)$ as follows:\\
\begin{eqnarray*}
\frac{1}{6}(2n^3 + 3n^2 + n) & \leq & n^3 \\
\frac{1}{6}(2n^3 + 3n^2 + n) & \geq & \frac{1}{6}n^3 \\
\end{eqnarray*}
with $n > 1$. Therefore, this function is $\Theta(n^3)$.
\end{enumerate}

\item
 Write a recurrence relation to compute the number of binary
strings with \textit{n} digits that do not have two consecutive
1's. Solve the recurrence, and determine what percentage of 8-bit
binary strings do not contain two consecutive 1's. \vskip 10pt
\textbf{ANSWER} In the following table, $T_n$ is the number of
binary numbers which do not have consecutive ones with $n$ binary
digits.
\begin{quote}
\begin{tabular}{r|l}
$n$ & $T_n$ \\
\hline
1 & 2   \\
2 & 3   \\
3 & 5   \\
4 & 8   \\
5 & 13  \\
\end{tabular}
\end{quote}
This can be expressed in the recurrence equation $T_n = T_{n-1} +
T_{n-2}$, with $T_1 = 2$ and $T_2 =3 $. This is just the Fibonacci
recurrence relation with initial conditions shifted over by 2,
since $F_3 = 2$, and $F_4 = 3$ That is,
\[
T_n = F_{n+2} =  \fr{1}{\sqrt{5}}\left[(\fr{1 +\sqrt{5}}{2})^{n+2}
- (\fr{1 + \sqrt{5} }{2})^{n+2}
 \right]
\]
The percentage binary numbers which do not have consecutive ones
for $n = 8$ is $\fr{55}{256}$, or $21.5\%$. Note, we don't need a
brute force method to figure out the recurrence relation. It can
be proven as follows: Either the string begins with a $0$, and
there are $T_{n-1}$ cases were this happens, or the string begins
with a $1$, so necessarilly, the first two bits are $10$, and
there are $T_{n-2}$ cases were this happens.

\item Strassen's algorithm shows how to multiply two \textit{n} by
\textit{n} matrices by multiplying 7 pairs of \textit{n/2} by
\textit{n/2} matrices, and then doing \textit{n}$^{\mathit{2}}$
operations to combine them. Write the recurrence equation for this
algorithm, and calculate the complexity of Strassen's algorithm,
by solving the recurrence by repeated substitution. \vskip 10pt
\textbf{ANSWER} Strassen's algorithm can be expressed in the
recurrence equation $T_n = 7T_\frac{n}{2} + n^2$.  Assuming $T_1 =
1$, we can solve this algorithm iteratively as follows:
\begin{eqnarray*}
T_1 & = & 1 \\
T_2 & = & 7 + 2^2 \\
T_4 & = & 7(7 + 2^2) 4^2 = 7^2 + 7\cdot 2^2 +  2^4 \\
T_8 & = & 7 ( 7^2 + 7\cdot 2^2 + 2^4 ) + 8^2 =  7^3 + 7^22^2 +
7\cdot2^4 + 2^6 \\
T_{2^k}  & = & 7^k + 7^{k-1}\cdot 2^2 + 7^{k-2} 2^4 + \cdots +
7\cdot 2^{2k-2} + 2^{2k}
\end{eqnarray*}
This is a geometric series with a ratio of $\frac{2^2}{7}$. Its
sum is
\[
\frac{2^{2k}\cdot 2^2 \cdot 7^{-1} - 7^k}{2^2 \cdot 7^{-1} -1} =
\frac{1}{3}(7^{k+1} - 2^{2k+2} )
\]
Because powers of seven beat powers of four, this is $\Theta (
7^{k+1} ) = \Theta ( 7^k) $. And since
\[
7^k = 2^{\log_2 7 \cdot k} = (2^k)^{(\log_2 7)} = n^{\log_2 7}
\]
So the algorithm is $\Theta ( n^{\log_2 7} ) $
\item
 Write and solve the recurrence
equations for the time complexity of the following recursive
algorithms. Explain why your equations are correct. \vskip 0.2 pt
\begin{enumerate}
\item
 To search for a value in a sorted list, compare it to the
middle value, and search the right half of the list if it is
larger,
and the left half if it is smaller.\\
\vskip 10pt  \textbf{ANSWER}
The equation for this algorithm is $T_n = T_\frac{n}{2} + 1$. It is $O(\log n)$\\

\item The maximum of a list of numbers is the larger of the maximum
of the first half and the maximum of the second half.\\
\vskip 10pt  \textbf{ANSWER}
The equation for this algorithm is $T_n = 2T_\frac{n}{2} + n$. It is $O(n \log n)$. \\

\item To sort a list of numbers, divide the list into four equal
parts. Sort each part. Merge these sorted four lists into two
sorted lists, and then merge the two into one. \vskip 10pt
\textbf{ANSWER}
 The equation for this algorithm is $T_n = 4T_\frac{n}{4} + 2n$. It is $O(n \log n)$. \\

\end{enumerate}

\item
 Solving the following recurrence by a change of variable:
\textit{a}$_{\mathit{n}}$ = 2\textit{a}\ensuremath{\sqrt{n}} + lg
\textit{n, a}$_{\mathit{1}}$ \textit{= 0.} (Solve by setting
\textit{m =} lg \textit{n}). You should solve this only when
\textit{n} is 2 to the power of 2$^{\mathit{k}}$. \vskip 10pt
\vskip 10pt \textbf{ANSWER}

Let $m = \log n \qquad \Rightarrow \qquad n = 2^m$ . Define $b_m =
a_n = a_{2^m}$. Then,
\begin{align*}
b_m = a_n & = 2 a_{\sqrt{n}} + \log n \\
          & = 2 a_{2^{\fr{m}{2}}} + m \\
          & = 2 b_{\fr{m}{2}} + m
\end{align*}
so
\begin{align*}
b_1 & = C \\
b_2 & = 2C + 1 = 2(C+1) \\
b_4 & = 2\cdot 2 (C + 1 ) + 2^2 = 2^2(C+ 2) \\
b_8 & = 2 \cdot 2^2 (C + 2) + 2^3 = 2^3 (C + 3) \\
    & \vdots \\
b_{2^k} & = 2^k (C + k)
\end{align*}
So
\[
a_{2^{2^k}} = 2^k ( C + k)
\]
or
\[
a_n = \log n ( C + \log \log n)
\]

\item Parenthesized Expressions

\begin{enumerate}
\item
A sequence of \textit{n+1} matrices
\textit{A}$_{\mathit{1}}$\textit{A}$_{\mathit{2}}$\textit{\dots
A}$_{\mathit{n+1}}$ can be multiplied together in many different
ways dependent on the way \textit{n} pairs of parentheses are
inserted. For example for \textit{n+1 = 3}, there are two ways to
insert the parentheses:
\textit{((A}$_{\mathit{1}}$\textit{A}$_{\mathit{2}}$\textit{)A}$_{\mathit{3}}$\textit{)
and
(A}$_{\mathit{1}}$\textit{(A}$_{\mathit{2}}$\textit{A}$_{\mathit{3}}$\textit{))}.
Write a recurrence equation for the number of ways to insert
\textit{k} pairs of parenthesis. Do not solve it. (Hint:
Concentrate on where the \textit{last} multiplication
occurs). \\
\vskip 10pt  \textbf{ANSWER} This algorithm can be expressed as
$$
T_n = T_1T_{n-1} + T_2T_{n-2} + T_3T_{n-3} + \ldots + T_{n-1}T_1.
$$
\item
Write a list of the different ways to parenthesize a sequence
of \textit{n+1} matrices for \textit{n+1=2,3,4.}\\
\vskip 10pt  \textbf{ANSWER} The following table shows the
possible combinations for $n+1 = 2,3,4$.
\begin{quote}
\begin{tabular}{r|l}
$n+1$ & possilibities\\
\hline
2   &   $(A_1 \cdot A_2)$ \\
\hline
3   &   $(A_1 \cdot (A_2 \cdot A_3))$   \\
    &   $((A_1 \cdot A_2) \cdot A_3)$   \\
\hline
4   &   $((A_1 \cdot (A_2 \cdot A_3)) \cdot A_4)$ \\
    &   $(((A_1 \cdot A_2) \cdot A_3) \cdot A_4)$ \\
    &   $(A_1 \cdot (A_2 \cdot (A_3 \cdot A_4)))$ \\
    &   $(A_1 \cdot ((A_2 \cdot A_3) \cdot A_4))$ \\
    &   $((A_1 \cdot A_2) \cdot (A_3 \cdot A_4))$ \\
\end{tabular}
\end{quote}
\item
A balanced arrangement of parenthesis is defined inductively as
follows:


The empty string is a balanced arrangement of parentheses. If \textit{x}
is balanced arrangement of parentheses then so is \textit{(x).} If \textit{u} and \textit{v} are
each a balanced arrangement of parentheses, then so is \textit{uv.}



Write a list of strings that represent a balanced arrangement
of \textit{n} parentheses for \textit{n=1,2,3.}\\
\vskip 10pt  \textbf{ANSWER}
 The following table shows the
possible combinations for $n= 1,2,3$.
\begin{quote}
\begin{tabular}{r|l}
$n$ & possilibities\\
\hline
1  &   $( )$ \\
\hline
2   &   $( ( ) )$   \\
    &   $( ) ( )$   \\
\hline
3   &   $( ( ) ) ( )$ \\
    &   $( ) ( ) ( )$ \\
    &   $( ( ( ) ) )$ \\
    &   $( ( ) ( ) )$ \\
    &   $( ) ( ( ) )$ \\
\end{tabular}
\end{quote}

\item
 Describe a 1-1 correspondence between the strings that are
balanced arrangements of \textit{n} pairs of parentheses, and the
number of ways to multiply a sequence of \textit{n+1} matrices.
\vskip 10pt  \textbf{ANSWER} It's easier to see this, if we map
both things to a third set of strings. For the pairs of
parentheses replace the left parentheses with $\alpha$ and the
right parentheses with $\beta$. For example:
\begin{quote}
\begin{tabular}{r|l| l}
$n$ & possilibities & transformed form\\
\hline
1  &   $( )$ & $\alpha \beta$ \\
\hline
2   &   $( ( ) )$  & $\alpha \alpha \beta \beta$ \\
    &   $( ) ( )$  & $\alpha \beta \alpha \beta$ \\
\hline
3   &   $( ( ) ) ( )$ & $\alpha \alpha \beta \beta \alpha \beta$ \\
    &   $( ) ( ) ( )$ & $\alpha \beta \alpha \beta \alpha \beta$ \\
    &   $( ( ( ) ) )$ & $\alpha \alpha \alpha \beta \beta \beta$ \\
    &   $( ( ) ( ) )$ & $\alpha \alpha \beta \alpha \beta \alpha$ \\
    &   $( ) ( ( ) )$ & $\alpha \beta \alpha \alpha \beta \beta$ \\
\end{tabular}
\end{quote}

For the ways to multiply $n+1$ matrices, remove the matrices and
the left parentheses from the expression, and replace any
multiplication by $\alpha$ and any right parenthesis by $\beta$.
For example:
\begin{quote}
\begin{tabular}{r|l|l}
$n+1$ & possilibities & transformed form\\
\hline
2   &   $(A_1 \cdot A_2)$ & $ \alpha \beta $\\
\hline
3   &   $(A_1 \cdot (A_2 \cdot A_3))$  & $\alpha \alpha \beta \beta $\\
    &   $((A_1 \cdot A_2) \cdot A_3)$  &  $\alpha \beta \alpha \beta $\\
\hline
4   &   $((A_1 \cdot (A_2 \cdot A_3)) \cdot A_4)$ & $\alpha \alpha \beta \beta \alpha \beta $\\
    &   $(((A_1 \cdot A_2) \cdot A_3) \cdot A_4)$ & $\alpha \beta \alpha \beta \alpha \beta $\\
    &   $(A_1 \cdot (A_2 \cdot (A_3 \cdot A_4)))$ & $\alpha \alpha \alpha \beta \beta \beta $\\
    &   $(A_1 \cdot ((A_2 \cdot A_3) \cdot A_4))$ & $\alpha \alpha \beta \alpha  \beta \beta $\\
    &   $((A_1 \cdot A_2) \cdot (A_3 \cdot A_4))$ & $\alpha \beta \alpha \alpha \beta \beta $\\
\end{tabular}
\end{quote}


\end{enumerate}

\item
Prove that any \textit{O({\textbar}E{\textbar})} time algorithm on
a planar graph is also \textit{O({\textbar}V{\textbar})}. (Hint:
Use the fact that every face has at least three vertices and
edges, and a counting argument, to calculate a relationship
between the number of faces and the number of edges. Then use
Euler's Theorem to derive a linear relationship between the number
of edges and the number of vertices.) \vskip 10pt  \textbf{ANSWER}
(The devil is in the details... Note that the statement is false
without the condition that the graph is simple, i.e. that there is
at most one edge connecting any two vertices. Otherwise we could
add an arbitrary number of edges between two vertices. and so make
the number of edges much much larger than the number or vertices )

The statement follows from the following claim: from page 505 of Rosen\\

On any simple connected planar graph with $e$ edges and $v$
vertices, $ e \le 3 v -6$.

Therefore if the algorithm satisfies $T_{Graph} \le C e$ for large
enough $e$, it satisfies
$$
T_{Graph} \le C ( 3 v -6 ) < 3 C v
$$
Hence the algorithm is $O( v ) $



\item
The following recurrence cannot be solved using the master
theorem. Explain why. Solve it directly by substitution, and
calculate its order of growth.
\[
T(n) = 4T(n/2) + (nlogn)^2 \]
and
\[
\textit{T(1) = 1.}
\]
\vskip 10pt  \textbf{ANSWER} It fails to satisfy the hypotheses of
the master theorem.. In this case $a= 4$ , $b =2$ and $\log_b^a =
2$\\
The function $n^2 \log n$ is $\Omega (n^{\log_b^a)} = \Omega
(n^2),$ however it is not polynomially greater than $n^2$ so the
master theorem does not apply. Let's solve this for $n = 2^k$ by
repeated substitution. The recurrence relation becomes
\[
T(2^k) = 4 T(2^{k-1}) + 4^k \cdot k^2
\]
Assuming $T(2^0) = 1$ Then,
\begin{eqnarray*}
T(2^1) & = & 4 + 4 \\
T(2^2) & = & (4^2 + 4^2) + 4^2 \cdot 2^2 \\
T(2^3) & = &  (4^3 + 4^2 + 4^3 \cdot 2^2) + 4^3 \cdot 3^2 \\
T(2^4) & = &  (4^4 + 4^4 + 4^4 \cdot 2^2 + 4^4 \cdot 3^2 ) + 4^4 \cdot 4^2 \\
& \vdots &   \\
T(2^k) & = &  4^k ( 1 +  1 + 2^2 + 3^2  + \cdots + k^2 ) \\
       & = &  4^k ( 1 +  \frac{k (k+1) (2k+1)}{6} ) \\
\end{eqnarray*}

\end{enumerate}
So $T(2^k)$ is $\Theta ( k^3 4^k ) $, or $\Theta( (\log n)^3 n^2
)$

\end{document}
