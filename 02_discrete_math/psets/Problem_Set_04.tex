\documentclass[12pt]{amsart}

\usepackage{fullpage}
\usepackage{psfig}
\newcommand{\fr}[2]{\frac{#1}{#2}}
\title[Problem Set 4]{Ars Digita University\\Month 2:  Discrete Mathematics\\Professor Shai Simonson\\Problem Set 4 Answers}

\begin{document}

\maketitle

\begin{enumerate}
\item {\bf Whats' wrong}
\begin{enumerate}
\item {\bf}
    The problem with the induction is that the base case is wrong.  If the base case choosen is where n = 1 then at the base case you can have a binary number of either 1 or 0 thus disproving that it is the same.


\item {\bf}
    The problem with this induction is again the base case fails because you can not reach the number 22.

\end{enumerate}


\item{\bf Second Problem}
\begin{enumerate}


\item {\bf Fibonacci Sequence}\\
\\
    Assumption:
    $$  F(n) = (\fr{1}{\sqrt {5}})[(\fr{1}{2} + \fr{\sqrt{5}}{2}^n) - (\fr{1}{2} -
        \fr{\sqrt{5}}{2})^n]$$
    Base Case:
$$
    F_1 = 1 \qquad    F_0 = 0
$$
        Prove:
    $$  F(n+1) = F(n) + F(n-1) \\$$
    Let $a = (\fr{1}{2} + \fr{\sqrt{5}}{2})$\\
    Let $b = (\fr{1}{2} - \fr{\sqrt{5}}{2})$\\
$$
               \fr{1}{\sqrt{5}}(a^{n+1} - b^{n+1}) =  \fr{1}{\sqrt{5}}(a^n - b^n) + \fr{1}{\sqrt{5}}
                (a^{n-1} - b^{n-1}))
$$
\\
        Divide both sides through by  $\sqrt{5}$\\
                    $$a^{n+1} - b^{n+1} =  a^n - b^n + a^{n-1} - b^{n-1}$$




\item {\bf The series}\\


    Assumption: $$1 + a + a^2 + a^3 \ldots + a^n = \fr{(1 - a^{n+1}}{1-a)}$$\\
    Base Case: $$n = f(1)$$\\
                     $$= \fr{(1 - a^2)}{1 - a} = \fr{((1 - a)(1 + a))}{(1 - a)} = 1 + a$$\\
                  $$ n = 1 + a$$

        Theory:    $$\fr{(1 - a^{n+1})}{1-a + a^{n+1}} = \fr{(1 - a^{n+2})}{1 -
        a}$$

                   $$(1 - a)(1 - a^{n+1})/1-a + (1 -a)(a^{n+2}) = (1 - a)(1 - a^{n+2})/1 -
                   a$$

                   $$(1 - a^{n+1}) + (1 - a)a^{n+1} = (1 -
                   a^{n+2})$$

           $$(1 - a^{n+1}) + a^{n+1} - a^{n+2} = (1 - a^{n+2})$$

           $$1- a^{n+2} = (1 - a^{n+2})$$
qed.
\\


\item {\bf The Mod Squad}
\\

    Assumption: $$4^{n+1} + 5^{2n-1} = 21a $$ where a is some constant
    $\geq $  0\\
  \\  Base Case:  $$n = 1$$
     $$4^{2} + 5^{1} = 21a$$
                     $$ 16 + 5 = 21a$$
                     $$ 21 = 21a$$

        Theory:     That for some n the assumption holds true.

                   $$ 4^{n+1} + 5^{2n-1} = 21b$$
                    $$4^{n+2} + 5^{2n+1} = 21b$$
                   $$ 4*4^{n+1} + 5^2 * 5^{2n-1} = 21b$$
                    \\    Let $4^{n+1} = 21a - 5^{2n-1}$ and substitute in
                       for
                        $4^{n+1}$\\
                   $$ 4*(21a - 5^{2n-1}) + 5^2 * 5^{2n-1} = 21b$$
                   $$ 4*21a - 4*5^{2n-1} + 5^2 * 5^{2n-1} = 21b$$
                   $$ 4*21a - 21* 5^{2n-1} = 21b$$
\\
        qed. since all terms are divisible by 21.



\item{\bf Binary Trees}
\\
       Assumption: The number of leaves in a complete binary tree is one more then the number of internal nodes.
       Base Case:  A complete tree of level 1 has 1 nodes and 2 leaves.

       Theory:  To increase from a tree of any depth of $l$ you must add two leaves to each old leaf.  This increase the
number of leaves by $2*l$ and turns all old leaves into nodes. You
now have twice as many leaves as before with an increase of nodes
by a factor of $2^{n-1}$ where $n$ is the depth at which you are
at. Since the previous relationship of nodes to leaves was $l = n
- 1.$ When you double the number of leaves and you add the number
of old leaves to the number of nodes.  The relationship stays the
same so the new tree still has $l = n - 1$ leaves.
\\
\item{\bf Disjointed Graphs}\\
    Assumption: That a graph with n edge-disjoint paths has n pairs of odd degree
vertices.
    Base Case: 1 odd pair graph has one edge.

    Theory: Take any graph of G with n + 1 pairs of odd vertices and
create an edge between them the graph would now have n odd
vertices.  Based on our assumption we know that this has n
disjoint paths.  If we remove the edge that we just placed on the
graph we now have an extra created n + 1 odd vertices and since we
removed this path we just created the need for another path to get
the new odd edge created.\\
\end{enumerate}






\item{\bf Recurrence Equations}\\
\begin{enumerate}
\item
$a_n = 6a_{n-1} - 8a_{n-2}, \qquad a_0 = 4, a_1 = 10$
$$ r^n = 6r^{n-1} - 8r^{n-2}$$
Divide through by $r^{n-2}$
$$ r^2 = 6r - 8$$
$$ r^2 - 6r - 8 = 0$$\\
$ (r - 2)(r - 4) $ giving roots of 2 and 4\\
$$ a_n = \alpha_1{2^n} - \alpha_2{4^n}$$
$$ a_0 = \alpha_1{2^0} - \alpha_2{4^0}$$
$$ a_1 = \alpha_1{2^1} - \alpha_2{4^1}$$
$$ a_n = 3({2^n}) + {4^n}$$

\item
$a_n = a_{n-1} + 2a_{n-2}, \qquad a_0 = 0, a_1 = 1$
$$ r^n = r^{n-1} - 2r^{n-2}$$
Divide through by $r^{n-2}$
$$ r^2 = r - 2$$
$$ r^2 - r - 2 = 0$$\\
$ (r - 2)(r + 1) $ giving roots of -1 and 2.\\
$$ a_n = \alpha_1{-1^n} + \alpha_2{2^n}$$
$$ a_0 = \alpha_1{-1^0} + \alpha_2{2^0}$$
$$ a_1 = \alpha_1{-1^1} + \alpha_2{2^1}$$
$$ a_n = \fr{-1}{3}({2^n}) + \fr{1}{3}{4^n}$$\\

\item
$a_n = 7a_{n-1} - 10a_{n-2} + 3^n, \qquad a_0 = 0, a_1 = 1$
$$a_n = 7a_{n-1} - 10a_{n-2} + p3^n$$

$$p3^n = 7p3^{n-1} + 10p3^{n-2} + 3^n$$
Divide through by $3^{n-2}$
$$p3^2 = 7p3 - 10p + 3^2$$
$$9p = 21p - 10p + 9$$
$$-2p = 9$$
$$ p = \fr{9}{-2}$$
\\

Substitute $\fr{9}{-2}$ in for p and get:
$$a_n = 7a_{n-1} - 10a_{n-2} + \fr{9}{-2}3^n$$
\\Solve for the Roots:\\
$$ r^n = 7r^{n-1} - 10r^{n-2}$$\\
Divide through by $r^{n-2}$
\\

$$ r^2 = 7r - 10$$
$$ r^2 - 7r - 10 = 0$$\\
$ (r - 5)(r - 2) $ giving roots of 2 and 5\\
$$ a_n = \alpha_1{5^n} - \alpha_2{2^n} + 3^n$$
$$ a_0 = \alpha_1{5^0} - \alpha_2{2^0} + 3^n$$
$$ a_1 = \alpha_1{5^1} - \alpha_2{2^1} + 3^n$$
$$ a_n = \fr{11}{6}({2^n}) - \fr{16}{6}{4^n}$$
\\

\item
$a_n = 3 - 6a_{n-1} - 9a_{n-2}, \qquad a_0 = 0, a_1 = 1$
$$a_n = 3 - 6a_{n-1} - 9a_{n-2} + p1^n$$

$$p1^n = 3 - 6p1^{n-1} + 9p1^{n-2}$$
Divide through by $1^{n-2}$
$$p1^2 = 3 - 6p + 9p$$
$$p = 3 - 6p + 9p$$
$$p = 3 -  6p - 9p$$
$$p = \fr{3}{16}$$\\





Substitute $\fr{3}{16}$ in for p and get:
$a_n = 3 - 6a_{n-1} - 9a_{n-2}, \fr{3}{16}1^n$
\\Solve for the Roots:\\
$$ r^n = - 6r^{n-1} + 9r^{n-2}$$

Divide through by $r^{n-2}$\\

$$ r^2 = - 6r + 9$$
$$ r^2 - 6r + 9 = 0$$\\
$ (r - 3)(r - 3) $ giving roots of -3 and -3\\
$$ a_n = \alpha_1{-3^n} - \alpha_2{-3^n} + \fr{3}{16}$$
$$ a_0 = \alpha_1{-3^0} - \alpha_2{-3^0} + \fr{3}{16}$$
$$ a_1 = \alpha_1{-3^1} - \alpha_2{-3^1} + \fr{3}{16}$$

$$a_n = \fr{-3}{16}-3^{n-1} - \fr{-1}{12}-3^{n-2} + \fr{9}{-2}3^n$$

\end{enumerate}


\item

\begin{enumerate}
\item


$$T(n) = n^2 + T(n-1)$$
$$T(n-1) = n^2 + (n-1)^2 + T(n-2)$$
$$T(n-2) = n^2 + (n-1)^2 + (n-2)^2 + T(n-3)$$

\item
$$T(n) = n^2 + T(n-1)$$
$$T(n) = T(n-1)$$

$$r^n = r^{n-1}$$
$$r = 1$$

$$T(n) = \alpha_1{1^n} + n(p_1n^2 + p_2n^1 + p_3n^0)1^n$$\\
$$p_1n^3 + p_2n^2 + p_3n^1 = n^2 + (n-1)(p_1(n-1)^2 + p_2(n-1)^1 +
p_3(n-1)^0)$$\\
$$p_1n^3 + p_2n^2 + p_3n = n^2 + p_1n^3 - 3p_1n^2 + 3p_1n + p_1 + p_2n^2 - 2p_2n + p_2  + p_3n -
p_3$$\\
$$0 = n^2(1  - 3p_1) + n(3p_1 - 2p_2) + (p_1 + p_2  - p_3)$$\\
$$p_1 = \fr{1}{2}   \qquad p_2 = \fr{1}{3}   \qquad p_3 = \fr{1}{6}
$$\\
$$T(n) = \alpha_1{1^n} + n(\fr{1}{3}n^2 + \fr{1}{2}n^1 +
\fr{1}{6})$$\\
$$T(0) = 0$$\\
$$T(n) = \fr{2n^3 + 2n^2 + n}{6}$$\\

\item

$$T(n) = \fr{2n^3 + 2n^2 + n}{6}$$
        When you have a recurrance equation that is a straight polynomial all you do is look to the largest degree and that is the $\Theta$ the
    degree.\\
\end{enumerate}
\item

$$T(n) = T(n-1) + T(n-2)$$\qquad T(1) = 2 \qquad T(2) = 3 \\
Notice this is a fibonacci series with this problem being n+2.
Substitute as appropriate.
$$ F(n) = (\fr{1}{\sqrt {5}})[(\fr{1}{2} + \fr{\sqrt{5}}{2})^{n+2} - (\fr{1}{2} -
        \fr{\sqrt{5}}{2})^{n+2}]$$\\

\item
$$$$
$$T(n)=7T(\fr{n}{2}) + n^2$$
$$T(\fr{n}{2})=7T(\fr{n}{2^2}) + (\fr{n}{2})^2$$
$$T(\fr{n}{4})=7T(\fr{n}{2^3}) + (\fr{n}{2^2})^2$$


$$T(n)=7(7T(\fr{n}{2^2}) + \fr{n}{2})^2 + n^2$$

$$T(n)=7^2(7T(\fr{n}{2^3}) + \fr{n}{2^2})^2 + (\fr{n}{2})^2 + \fr{n}{2})^2 + n^2 + n^2$$
$$T(n)=7^kT(\fr{n}{2^k} + n^2\fr{1-(\fr{7}{4})^k}{1-(\fr{7}{4}}$$
$$k = log_2n$$
$$T(n)=7^(log_2n)T(\fr{n}{2^(log_2n)} +
n^2\fr{1-(\fr{7}{4})^(log_2n)}{1-\fr{7}{4}}$$\\
\item
\begin{enumerate}
\item
$$$$
$$T(n)=T(\fr{n}{2}) + C$$
    According to the Master's Theorem the order of complexity is
    $log_2(n)$.\\

\item
$$$$$$T(n)= 2T(\fr{n}{2}) + 1$$
    According to the Master's Theorem the order of complexity is
    $n$.\\

\item
$$$$$$T(n)= 4T(\fr{n}{4}) + 2(\fr{n}{4} + \fr{n}{4}) + (\fr{n}{2}\fr{n}{2})$$
    According to the Master's Theorem the order of complexity is
    $nlog_2(n)$.\\

\end{enumerate}

\item
\item
a b c d\\
a b c d\\
a b c d \\
a b c d \\
a b c d \\
a b c d\\



\end{enumerate}
\end{document}
