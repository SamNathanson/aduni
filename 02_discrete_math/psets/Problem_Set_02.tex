%&LaTeX
\documentclass{article}
                                                                                                    
                                                                                                    
                                                                                                    
                                                                                                    
                                                                                                    
                                                                                                    
                                                                                                    
                                                                                                    
                                                                                                    
                                                                                                    
\newcommand{\tab}{\makebox[4em]{}}


\begin{document}


\begin{center}
\textbf{ArsDigita University}


\end{center}

\begin{center}
\textbf{Month 2: Discrete Mathematics - Professor Shai Simonson}


\end{center}

\begin{center}
\textbf{Problem Set 2 -- Sets, Functions, Big-O, Rates of Growth}


\end{center}

1.\tab 
Prove by formal logic:\\
a.\tab 
The complement of the union of two sets equals the intersection 
of the complements.\\
b.\tab The complement of the intersection of two sets equals the union 
of the complements.\\
c.\tab (\textit{B} \ensuremath{-}\textit{A}) \ensuremath{\cup} (\textit{C} \ensuremath{-} \textit{A}) = (\textit{B} \ensuremath{\cup} \textit{C}) 
-- \textit{A}. \\
d.\tab If two sets are subsets of each other then they are equal.


2.\tab 
Generalize De Morgan's laws for \textit{n} sets and prove the laws 
by induction.


3.\tab 
Prove by induction on the size of the set, that the power set \textit{P(A)} has 
cardinality \textit{2}$^{\mathit{{\textbar}A}}$$^{{\textbar}}$.


\textit{4.\tab }
\textit{A} \ensuremath{\oplus} \textit{B} is defined to be the set of all elements 
in \textit{A} or \textit{B} but not in both \textit{A} and \textit{B.} \\
a.\tab 
Determine whether or not \ensuremath{\oplus} is commutative. Prove your 
answer. \\
b.\tab Determine whether \ensuremath{\oplus} is associative. Prove your answer.\\
c.\tab Determine whether \ensuremath{\oplus} can be distributed over union. 
Prove your answer.\\
d.\tab Determine whether \ensuremath{\oplus} can be distributed over intersection. 
Prove your answer.


5.\tab 
Assume a universal set of 8 elements. Given a set \textit{A = a}$_{\mathit{7}}$\textit{a}$_{\mathit{6}}$\textit{a}$_{\mathit{5}}$\textit{a}$_{\mathit{4}}$\textit{a}$_{\mathit{3}}$\textit{a}$_{\mathit{2}}$\textit{a}$_{\mathit{1}}$\textit{a}$_{\mathit{0}}$\textit{,} 
represented by 8 bits, explain how to use bitwise and/or/not 
operations, in order to:\\
a.\tab 
Extract the rightmost bit of set \textit{A}.\tab b. Extract the odd numbered 
bits. 


c.\tab Make bits 4-6 equal to 1.


Given another set \textit{B,} explain how to:\tab a. Determine if A \ensuremath{\subseteq} 
B.\tab b. Extract A \ensuremath{-} B.


6.\tab 
Given a set of 29 students, where 8 need housing and financial 
aid; 12 need housing, financial aid, and an e-mail account; 17 
need an e-mail account and financial aid; 23 need housing, 26 
need financial aid, 19 need e-mail accounts, and 4 students don't 
need anything. How many students need both housing and financial 
aid?


7.\tab 
How many numbers are there between 1 and 10000, which are either 
even, end in 0, or have the sum of its digits divisible by 9? 
Hint: Use inclusion/exclusion.


8.\tab 
Prove that \textit{f(x)} = \textit{x}$^{3}$ -- 1000\textit{x}$^{2}$ + \textit{x} -- 1 is \ensuremath{\Omega}(\textit{x}$^{\mathit{3}}$\textit{)} 
and O(\textit{x}$^{\mathit{3}}$).


9.\tab 
Prove that 1$^{k}$ + 2$^{k}$ + 3$^{k}$ + \dots  + n$^{k}$ is \ensuremath{\Omega}(\textit{n}$^{\mathit{k+1}}$\textit{)} 
and O(\textit{n}$^{\mathit{k+1}}$), for any constant \textit{k.} 


10.\tab 
Order the following functions in order of their growth rate. 
If \textit{f(x)} is O(\textit{g(x))}, but \textit{g(x)} is not O(\textit{f(x)),} then 
put \textit{f(x)} above \textit{g(x).} If they are each big-O of each other, 
then place them on the same level.



\textit{x}$^{\mathit{2}}$ \textit{+ x}$^{\mathit{3}}$ \tab \tab \textit{3}$^{\mathit{x}}$\tab \textit{x!\tab x/log}$_{\mathit{2}}$\textit{x 
\tab x}$^{\mathit{2}}$\textit{+ 2}$^{\mathit{x}}$\tab \tab \tab \textit{xlog}$_{\mathit{2}}$\textit{x\tab \tab 2}$^{\mathit{xlogx}}$\tab \textit{logx}$^{\mathit{2}}$\\
\textit{log}$_{\mathit{2}}$\textit{x!\tab \tab log}$_{\mathit{2}}$\textit{x,\tab lnx,\tab 2}$^{\mathit{x\tab \tab }}$\textit{x(1+2+\dots + 
x)\tab \tab loglogx\tab \tab 2}$^{\mathit{x}}$$^{\mathit{2}}$\tab $^{\mathit{log}}$$^{\mathit{2}}$$^{\mathit{x}}$

11.\tab 
Compute the sum of the infinite series below.\\
a.\tab 
1 + 1/4 + 1/16 + 1/64 + \dots \\
b.\tab 1 + 2/4 + 3/16 + 4/64 + \dots 


12.\tab 
 Telescoping Series.\\
a.\tab 
Using the identity 1/(\textit{k)(k+1) =} 1\textit{/k --} 1\textit{/(k+1),} find 
the value of the infinite sum 1/(1*2) + 1/(2*3) + 1/(3*4) + \dots .\\
b.\tab The \textit{n}th triangle number is defined to be the sum of the 
first \textit{n} positive integers. What is the value of the infinite 
sum of the reciprocals of the triangle numbers.


13.\tab 
Countability Proofs.\\
a.\tab 
Prove that the positive odd numbers have the same cardinality 
as the positive even numbers.\\
b.\tab Prove that the set of all integer points in the positive quadrant 
of 3-dimensional space are countable. \\
c.\tab Prove by induction that the set of all vectors in \textit{k} dimensions 
with positive integer values is countable.\\
d.\tab Prove that the number of Scheme programs is countable. Hint: 
Order them by the number of characters each contains.


14.\tab 
The following is a version of Russell's Paradox for sets, described 
in your text (Rosen) in problem 26, on page 45. Consider any 
computer program that takes other programs as input, and outputs \textit{yes} or \textit{no} 
based on some criterion. (A compiler or interpreter, for example). 
It is possible for such a program to be fed back into itself, 
and depending on the program it might either say \textit{yes} (a Scheme 
interpreter written in Scheme) or \textit{no} (a Scheme interpreter 
written in Java). Call the programs that say \textit{no} to themselves \textit{self-hating} programs. 
Now assume that there is a computer program that takes other 
programs as input and says \textit{yes} to all the \textit{self-hating} 
programs. \\
a.\tab 
Assuming this program never runs forever on an input, analyze 
what happens when this program is input to itself. \\
b.\tab Consider the possibility that the program will run forever 
answering neither \textit{yes} or \textit{no} on some input\textit{.}


15.\tab 
Counting each arithmetic calculation or comparison, extraction 
or exchange of a card as one operation, what is the worst-case 
order of growth of an algorithm that sorts numbered cards in 
the following way? 


a.\tab 
Find the largest valued card in the deck by shuffling through 
one card at a time extracting a card if it is the largest one 
seen so far, and swapping the previously largest card back into 
the deck. When the largest has been found, place this card face 
down in a new pile and repeat the previous process until no cars 
in the original pile are left. Explain your answer.\\
b.\tab This time we assume that the largest number on any of the \textit{n} cards 
is \textit{n}$^{\mathit{2}}$\textit{.} We sort the cards by placing a set of \textit{n}$^{\mathit{2}}$ 
cards numbered from 1 to \textit{n}$^{\mathit{2}}$ on a table. Then one by 
one, place each card on top of the number equal to it on the 
desk. This method can be improved to work in linear time. Explain 
how. Hint: Use division.

\end{document}
