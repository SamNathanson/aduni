
\documentclass{article}

\setlength{\textwidth}{6.0in}
\setlength{\textheight}{9.0in}
\setlength{\headheight}{0.0in}
\setlength{\headsep}{0.0in}
\setlength{\parskip}{1.5ex plus 0.5ex minus 0.5ex}
\setlength{\topmargin}{0.5in}
\setlength{\topskip}{0.0in}
\setlength{\evensidemargin}{0.5in}
\setlength{\oddsidemargin}{0.5in}
\newcounter{exercise}
\newcounter{problem}
\newcounter{question}[problem]
\newcommand{\exercise}[1]{\bigskip\noindent{\large\bf Exercise \stepcounter{exercise} \arabic{exercise}: {#1}}}
\newcommand{\problem}[1]{\bigskip\noindent{\large\bf Problem \stepcounter{problem} \arabic{problem}: {#1}}}
\newcommand{\question}[1]{\pagebreak[3]\bigskip\indent{\bf Question} \stepcounter{question} \arabic{question}: {#1}? }


\begin{document}

\newcommand{\answertab}[4]{ 
\par
\begin{tabular}{l l}
A. {#1} & B. {#2} \\
C. {#3} & D. {#4}
\end{tabular}
\par
\medskip
Final Answer: 
\medskip
}

\begin{center}
\LARGE
Object-Oriented Programming in Java
\end{center}
\bigskip
{\noindent\Large\bf Quiz 1 \hfill  Jan 10, 2001}
\hrule

\problem{Who wants to be a Java developer?}
\noindent (with apologies to Regis)

Fill in your answer in the space provided.


\question
{Which is these word-pairs represents the IS-A relationship ( in English)}

\answertab{duck/bird}{bird/fly}{beak/bird}{bird/nest}

\question
{Which is these pairs represents the HAS-A relationship}

\answertab{bird/duck}{bird/fly}{bird/beak}{bird/flying\_animal}

\question
{Which is these pairs represents the BEHAVES-LIKE relationship}

\answertab{bird/duck}{bird/fly}{bird/beak}{bird/flying\_animal}

\question
{Which phrase below best describes the relationship between Country and
Capital}

\answertab{IS-A}{HAS-A}{USES}{BEHAVES-LIKE}


\question
{Which Java technology best supports the IS-A relationship}

\answertab{Inheritance}{Access Restriction}{Strong Typing}{Garbage Collection}

\question
{Which Java keyword is used to specify inheritance}

\answertab{extends}{implements}{public}{static}

\question
{Which Java keyword is used to specify compliance with an interface}

\answertab{extends}{implements}{public}{static}

\question
{Which Java keywords can not appear in a class declaration}

\answertab{extends}{private}{void}{abstract}

\question
{Is it possible to use the techniques of object oriented programming
in a language that does not specifically support OOP}

\answertab{Yes}{No}{}{}


\question
{Are private classes only for malicious corporations closely guarding
their proprietary code}

\answertab{Yes}{No}{}{}


\question
{Which Java keywords can not appear in instance variable definitions}

\answertab{extends}{protected}{int}{private}

\question
{Which Java keywords can not appear in method definitions}

\answertab{implements}{void}{static}{private}


\question
{Which of these fragments represents the HAS-A relationship between
Foo and Bar}

\answertab
{class Foo extends Bar\{\}
}
{
class Foo implements Bar\{
\}
}
{
class Foo \{
   private Bar mybar;
\}
}
{
abstract class Bar extends Foo\{
\}
}


\question
{Which line of code in the following can be inserted in place of the comments,
to perform the initialization described by the comments}
\begin{verbatim}
     public class T {
          int r;
          int s;

          T(int x, int y) {
               r = x;
               s = y;
          }
     }

     class S extends T {
          int t;

          public S(int x, int y, int z){
               // insert here the code
               // that would do the correct initialization
               // r= x  and s= y
               t=z;
          }
     }
\end{verbatim}

\answertab{T(x, y);}{this(x, y);}{super(x, y);}{super(x, y, z);}

\question
{What do the  'public' and 'private' keywords relate to}
\answertab{Typing}{Garbage Collection}{Polymorphism}{Access Restriction}

\question
{Which Java technologies  most powerfully supports the programming
concept of ``encapsulation''}

\answertab{Container Classes}{Access Restriction}{Garbage Collection}{Typing}

\question
{Which Java technologies  most powerfully support the programming
concept of ``abstraction''}

\answertab{Inheritance and Polymorphism}{Heap Allocation}{Garbage Collection}{Applets}


\newpage
\problem{Program Analysis: Short Answer}

\question
{What are the values of {\tt i},{\tt a}, and {\tt b}
after the following code fragment runs.}
\begin{verbatim}
int i;
int a=1;
int b=0;
for(i=0;i<10;i++){
   a = a * 2;  
   b++;
}
\end{verbatim}
\par
{\tt i =  }
\par
{\tt a = }
\par
{\tt b = }

\par

\question
{What is the value of {\tt a}
after the following code fragment runs}
\begin{verbatim}
int a=0;
while(true){
    if(a > 20) break;
    a = a+1;
}
\end{verbatim}
\par
{\tt a =  }
\par

\question
{What three lines will the following code print out
when compiled and run}

\begin{verbatim}
public class Spindle{
        public static void main(String[] args){
                int[] a = new int[] {1,2,3,4};
                int b = 5;
                int c =6;

                Fold.mutilate( a , b, c);
                System.out.println( a[0]);
                System.out.println( b);
                System.out.println( c );
        }

}

class Fold{
        static void mutilate( int[] a, int b , int c) {
                a[0] = 7;
                b = 8;
                c = 9;
        }
}
\end{verbatim}

line1:

line2:

line3:


\question
{What does the class Test below print out if it is compiled and run}

\begin{verbatim}
public class Test
{
        int x;

        Test(String s){
                this();
                System.out.println(s);
                System.out.println("This");
        }

        Test(int x){
                this("is");
        }

        Test(){
                System.out.println("a test.");
        }

        public static void main(String[] args)
        {
           int val = 2;
           Test t = new Test(val);
        }
}
\end{verbatim}

Test prints out:
\vspace{1.0in}

\newpage
The next three short answer question use the following group of classes.

\begin{verbatim}

/**
 * An interface with one method
 */
interface Face1{
   public void bar();
}

/**
 * A class with an accessor and method foo()
 */
class Class1{
    private int size = 0;
    private String name;
    
    Class1(String name){
       this.name = name;
    }

    public String getName(){
       return(name);
    }

    public void foo(){
       System.out.println(``Class1.foo()'');
    }
}

/**
 * A class inheriting from Class1
 */
class Class2 extends Class1 implements Face1{
    int size = 0;
    int x;

    Class2(String name){
       super(name);
    }

    public void foo(){
       System.out.println(``Class2.foo()'');
    }

    public void bar(){
       System.out.println(``Class2.bar()'');
    }
}

\end{verbatim}

\newpage
\question
{What does the following print out}
\begin{verbatim}
public class TestClass{

   public static void main(String[] args){
       Class1 c = new Class2(``me'');
       c.foo();
   }

}
\end{verbatim}
\par
Ans:
\par

\question
{What does the following print out}
\begin{verbatim}
public class TestClass{

   public static void main(String[] args){
        Class1 c = new Class1(``me'');
        c.foo();
   }

}
\end{verbatim}
\par
Ans:
\par

\question
{What does the following print out}
\begin{verbatim}
public class TestClass{

   public static void main(String[] args){
        Class2 c = new Class2(``me'');     
        System.out.println(c.getName());
   }
}
\end{verbatim}
\par
Ans:
\par

\question
{What does the following print out}
\begin{verbatim}
public class TestClass{

   public static void main(String[] args){
        Face1 c = new Class2(``me'');     
        c.bar();
   }
}
\end{verbatim}
\par
Ans:
\par


\newpage
\problem{Class Implementation}

In this problem we ask you to implement part of a class Fraction, which
represents fractions of the form $a/b$ where $a$ and $b$ are integers,
($1/2$, 4/6, 101/432. etc.).
In the space below, complete the class Fraction. Use {\tt int}
to store the numerator and denominator.

Include a constructor that initializes from a numerator and denominator.

Write an instance method for addition.
[Hint: $a/b + c/d = (ad+bc)/bd$.]

Write a static method for multiplication.
[Hint: $a/b * c/d = ac/bd$.]

Write a static {\tt main()} that allocates two Fractions,
$1/2$ and $3/4$ and stores their sum in a third variable.

(Be merciful to graders: please write clearly).

\begin{verbatim}
/**
 * Class Fraction. Represent the fraction X/Y where X and Y are integers
 */
public class Fraction{
   /**
    * Your instance vars here
    */









   /**
    * Your constructor here
    */















Continued on next page







   /**
    * Your add method here
    */
















   /**
    * Your static mult method here
    */














   /**
    * Your static main  here
    */









}
\end{verbatim}



\newpage
\problem{OOP Design}

You are designing an OO program to help people plan an evening out.
It is proposed that the following classes be used. Add inheritance
( {\tt extends} phrases ) and instance variables to show the
IS-A and HAS-A relationships between these classes. Do not introduce
any additional classes or and instance variables of any basic type.
[Note: By Business, we mean anything that would appear in the Yellow Pages].

\begin{verbatim}
class Address
{



}

class Business
{





}

class City
{





}

class Restaurant
{





}

class Theater
{






}
\end{verbatim}

\end{document}





