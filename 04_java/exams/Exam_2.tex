
\documentclass{article}

\setlength{\textwidth}{6.0in}
\setlength{\textheight}{9.0in}
\setlength{\headheight}{0.0in}
\setlength{\headsep}{0.0in}
\setlength{\parskip}{1.5ex plus 0.5ex minus 0.5ex}
\setlength{\topmargin}{0.5in}
\setlength{\topskip}{0.0in}
\setlength{\evensidemargin}{0.5in}
\setlength{\oddsidemargin}{0.5in}
\newcounter{exercise}
\newcounter{problem}
\newcounter{question}[problem]
\newcommand{\exercise}[1]{\bigskip\noindent{\large\bf Exercise \stepcounter{exercise} \arabic{exercise}: {#1}}}
\newcommand{\problem}[1]{\bigskip\noindent{\large\bf Problem \stepcounter{problem} \arabic{problem}: {#1}}}
\newcommand{\question}[1]{\pagebreak[3]\bigskip\indent{\bf Question} \stepcounter{question} \arabic{question}: {#1}? }


\begin{document}

\newcommand{\answertab}[4]{ 
\par
\begin{tabular}{l l}
A. {#1} & B. {#2} \\
C. {#3} & D. {#4}
\end{tabular}
\par
\medskip
Your Answer: 
\medskip
}

\begin{center}
\LARGE
Object-Oriented Programming in Java
\end{center}
\bigskip
{\noindent\Large\bf Quiz 2 \hfill  Jan 24, 2001}
\hrule

\problem{Multiple Choice}
(Write letter corresponding to your choice in space provided.)

Fill in your answer in the space provided.

\question
{In Java, Color, JButton, and Exception are all what}

\answertab{widgets}{frames}{events}{objects}

\question
{Menus, buttons, and text fields are all what}

\answertab{handlers}{streams}{events}{widgets}

\question
{Which of the following classes is not commonly used in File Processing}

\answertab{OutputStream}{StringTokenizer}{ActionAdapter}{PrintWriter}

\question
{Which of the following classes must be extended to be used}

\answertab{AbstractAction}{InputStream}{Math}{StringTokenizer}

\question
{Which of the following method types cannot return an error code}

\answertab{accessor}{private}{constructor}{synchronized}

\question
{Which of the following keywords is NOT used in exception handling}

\answertab{try}{catch}{throw}{final}

\question
{Which of the following would most likely be used in a program that
download from the URL {\tt ftp://104.200.34.0:8080/SameGame.jar}}

\answertab
{{\tt new Socket(8080,"104.200.34.0")}}
{{\tt new ServerSocket(8080)}}
{{\tt new BufferedReader("104.200.34.0",8080)}}
{{\tt new Socket("104.200.34.0",8080)}}



\question
{In Program 1 (page 3 of quiz), the line {\tt super.paintComponent()} does
what}
\answertab
{Calls the {\tt paintComponent()} of JPanel }
{Calls the {\tt paintComponent()} of MyFrame }
{Calls the {\tt paintComponent()} of contentpane }
{Draws the window nicer than usual}

\question{In Program 1, what color is the rectangle after the 
button is clicked}
\answertab
{red}
{blue}
{green}
{black}

\question{In Program 1, what color is the rectangle initially}
\answertab
{red}
{blue}
{green}
{black}

\question{In Program 1, which of following MUST be an inner class
as implemented in the code below} (In other words, which if the following
would not work if moved verbatim outside of MyPanel?)
\answertab
{{\tt ButtonHandler}}
{{\tt MouseHandler}}
{Both A. and B.}
{Neither A. nor B.}

\question{In Program 1, what is {\tt Color.red}}
\answertab
{a public static variable}
{a public instance variable}
{a public static method}
{a public instance method}


\question{In Program 1, what is {\tt paintComponent} (in MyPanel)}
\answertab
{an inner class}
{a public static method}
{an event listener}
{a public instance method}

\question{Can event-based programming be implemented without multi-threading}
\answertab
{Yes}
{No}


\newpage
{\large\bf Program 1}
{\small
\begin{verbatim}

class MyFrame extends JFrame{
    public MyFrame(){
        setTitle("WidgetTest");
        setSize(300,300);        // size in pixels
        setDefaultCloseOperation(JFrame.EXIT_ON_CLOSE);
    }
}

class MyPanel extends JPanel{
    Color current = Color.red;
    JButton button = new JButton("Button");

    class ButtonHandler implements ActionListener{
        public void actionPerformed(ActionEvent e){
            current = Color.green;
            System.out.println("Mike");
            repaint();
        }
        
    }

    class MouseHandler extends MouseAdapter{
        public void mouseClicked(MouseEvent ev){
            if(ev.getX() <= 100) System.out.println("Rusty");
            if(ev.getY() >= 100) current = Color.blue;
            System.out.println("Dimitri");
            repaint();
        }
    }

    MyPanel(){
        button.addActionListener(new ButtonHandler());
        addMouseListener(new MouseHandler());
        add(button);
       System.out.println("Alan");
    }

    public void paintComponent(Graphics g){
        super.paintComponent(g);
        Graphics2D g2 = (Graphics2D)g;
        g.setColor(current);
        g2.fill(new Rectangle2D.Float(0,100,100,100));
    }

}

public class WidgetTest{
    public static void main(String[] args){
        MyFrame myframe = new MyFrame();
        Container contentpane = myframe.getContentPane();
        contentpane.add(new MyPanel());
        myframe.show();
    }
}
\end{verbatim}
}
\newpage


{\large\bf Program 2}
{\small
\begin{verbatim}

import java.io.*;

class MyException extends Exception{
    int value;
    MyException(int val){
       value = val;
    }
    
    public int getValue(){
       return(value);
    }
}

class TestClass2{

    public static void foo(int x) throws MyException,IOException{
       if(x < 0){
           System.out.println("X too low");
           throw(new MyException(x));
       }
       if(x > 100){
           System.out.println("X too high");
           throw(new IOException("Mike"));
       }
       System.out.println("X OK");
    }

    // Test main, convert argument to int can call test.
    public static void main(String[] args){
       int x = Integer.parseInt(args[0]); // convert first argument to int
       // 
       try{
           foo(x);
           System.out.println("Mike");
       }
       catch(IOException e){
           System.out.println("Rusty");
       }
       catch(MyException e){
           System.out.println("Dimitri");
           System.out.println(e.getValue());
       }
    }
}
\end{verbatim}
}
\newpage

\problem{Program Analysis: Short Answer}

\question{What does Program 1 print when it starts up}
\par
Answer:
\bigskip

\question{What does Program 1 print when the mouse is clicked
inside the filled rectangle}
\par
Answer:
\bigskip

\question{What does color is the filled rectangle drawn by Program 1
after the mouse is clicked inside it}
\par
Answer:
\bigskip

\question{If the line {\tt button.addActionListener(new ButtonHandler());}
is deleted from Program 1 (and the program is recompiled) what will
it print when the button is clicked}
\par
Answer:
\bigskip

\question{What does Program 2 (see previous page) print out when called as 
{\tt java TestClass2 2}}
\par
Answer:
\vspace{2.0in}

\question{What does Program 2 print out when called as 
{\tt java TestClass2 200}}
\par
Answer:
\vspace{2.0in}

\question{What does Program 2 print out when called as 
{\tt java TestClass2 -200}}
\par
Answer:
\vspace{2.0in}

The next questions refer to the Account class in an imaginary
banking application shown below.
When answering the questions, consider the following
scenario. There are two Account instances, {\tt savings} and
{\tt checking}. There are several threads sharing these
instance, some threads transferring from {\tt savings} to
{\t checking}, some transferring from {\tt checking} to {\tt savings}.
\begin{verbatim}
class Account{
    int amount = 1000;  // always start with 1000
    
    public void deposit(int x){
       amount = amount + x;
    }
    public void withdraw(int x){
       amount = amount - x;
    }

    public int balance(){
       return(amount);
    }
    public void transfer(int x, Account dest){
       withdraw(x);    // withdraw x from this account
       dest.deposit(x); // deposit in transfer account
    }

}
\end{verbatim}

\question{Is the class implementation as written thread-safe} Meaning,
can any number of threads call any sequence of methods on shared instances
and produce a deterministic predictable result regardless of scheduling policy.
\par
Answer:
\vspace{1.5in}

\question{If the {\tt deposit()} and {\tt withdraw} methods
are {\tt synchronized}, is the class implementation thread-safe}
\par
Answer:
\vspace{1.0in}

\question{If the {\tt transfer()} method ONLY
is {\tt synchronized}, is the class implementation thread-safe}
\par
Answer:
\vspace{1.0in}

\question{If ALL methods in the class are synchronized will this
class work properly? If not, why not}
\par
Answer:
\vspace{3.0in}



\end{document}







